\section{Results}

\subsection{Comparison of sequence collections at "study"-level}
Of the 4340 OTU centroids established for this study (includes OTUs from \citet{Steven_2013} and \cite{Garcia_Pichel_2013}) 445 and 870 have matches in the Living Tree Project (LTP) a collection of 16S gene sequences for all sequenced type strains \cite{Yarza_2008} (LTP version 115).

Similar numbers of total OTUs were found in each data set explored in this study (i.e. the DNA-SIP data presented here, the data presented by \citet{Steven_2013} and by \citet{Garcia_Pichel_2013}). Specifically, there were 3079 OTUs (209354 total sequences after quality control) in the DNA-SIP data, 3203 OTUs (129033 total sequences after quality control) in the \citet{Garcia_Pichel_2013} study, and 2481 OTUs (129358 total sequences after quality control) in the \citet{Steven_2013} study.

There is a stark contrast in the total percentage of sequences annotated as \textit{Firmicutes} between the raw environmental samples and the DNA-SIP data. \textit{Firmicutes} represent only 0.21\% and 0.23\% of total phylum level sequence annotations in the \citet{Steven_2013} and \citet{Garcia_Pichel_2013} study, respectively. In the DNA-SIP sequcuence collection, \textit{Firmicutes} make up 19\% of phylum level sequence annotations. \textit{Cyanobacteria} and \textit{Proteobacteria} were the top two phylum annotations for all three studies but only the DNA-SIP data had more \textit{Proteobacteria} annotations than \textit{Cyanobacteria}.

The DNA-SIP data set shares more OTUs with the \citet{Steven_2013} (56\% of total OTUs found in either of the two data sets) than it does with the \citet{Garcia_Pichel_2013} data (46\% of total OTUs between both data sets). The \cite{Steven_2013} and \cite{Garcia_Pichel_2013} share 46\% of total OTUs between the two sequence collections.

\subsection{Ordination of CsCl gradient fraction SSU rRNA libraries}
Ordination of Bray-Curtis CITE distances between CsCl gradient fraction sequence libraries with principal coordinates analysis CITE shows the labelled gradient fraction libraries diverge from control in the heavy fractions (Figure X). When the labelled and control gradient fractions are paired such that each pair contains a control fraction and labelled fraction with a density difference below X.XX g/mL, the Bray-Curtis distance between the fraction pair is proportional to the density of the labelled fraction (p-value 0.00052, r$^{2}$ 0.3315) (Figure X).

adonis (~label v control, heavy fractions)

include figure depicting paired fraction distance with density (p-value, correlation)

\subsection{Identities of possible $^{15}$N incorporators}

l2fc figure

LTP BLAST table

\subsubsection{Diversity of Clostridium $^{15}$N incorporators}

Clostridium tree

\subsubsection{Diversity of Gamma-proteobacterial $^{15}$N incorporators}

Discuss gamma-proteos

Diazotrophs from DNA-SIP in "raw" datasets

Heterocystous cyanobacteria in "raw" datasets



