\section{Results}
\subsection{Ordination of CsCl gradient fraction SSU rRNA sequence collections
shows heavy fractions from control and labeled CsCl gradients are
different} 
BSC were incubated for 4 days in the presence or absence of $^{15}$N$_{2}$ and
DNA was extracted for DNA-SIP at 2 and 4 days. Fractionation of CsCl gradients
permitted separation of DNA on the basis of buoyant density. Ordination of
Bray-Curtis \citep{Bray_1957} distances between gradient fractions (based on OTU
abundance within each fraction) reveals that labeled gradient fractions (i.e.
gradient fractions from gradients with $^{15}$N$_{2}$ labeled DNA) diverge from
control (i.e. DNA from incubations without $^{15}$N$_{2}$) at the ``heavy`` end
of the CsCl gradients (Figure~\ref{fig:ord_heavy} and
Figure~\ref{fig:ordination}). Bray-Curtis distances between heavy gradient
fractions are consistent label/control groups (p-value: 0.001, r$^{2}$: 0.18, 
Adonis test \citep{Anderson_2001}). 
\subsection{OTUs responsive to $^{15}$N$_{2}$ are primarily \textit{Proteobacteria}
and \textit{Clostridiaceae}}
A statistically significant increase in OTU abundance in heavy fractions of
$^{15}$N$_{2}$ labeled samples relative to corresponding control fractions
provides evidence for OTUs that have incorporated $^{15}$N into their DNA.
Specifically, we compared OTU proportion means between labeled and
control samples from heavy gradient fractions using statistics developed to
find differentially expressed genes with RNASeq data \citep{24699258,
Love_2014}. OTUs that incorporated $^{15}$N into DNA and increased in bouyant
density were identified by rejecting the the null hypothesis that the labeled
versus control proportion mean ratio for an OTU (considering only
heavy fractions) was below a chosen
threshold (see methods). p-values were adjusted by the BH method
\citep{citeulike:1042553} and we used a false discovery rate (FDR) cutoff of
0.10 (typical FDR threshold in gene expression data analysis). A total of 2,127
and 2,160 OTUs were detected in days 2 and 4, respectively, and interrogated
for evidence of $^{15}$N$_{2}$-labelling. Of these OTUs, only 208
and 233, respectively, passed a sparsity threshold we applied as an independent
filtering step to pre-screen out OTUs not likely to produce significant
p-values (see \citet{Love_2014} for discussion of independent filtering). Of OTUs
passing sparsity criteria 38 were found to be enriched significantly in
``heavy'' fractions relative to control. These OTUs likely incorporated
$^{15}$N into DNA ($^{15}$N$_{2}$ ``responders''). Of these 38, 26 are
annotated as \textit{Firmicutes}, 9 as \textit{Proteobacteria}, 2 as
\textit{Acidobacteria} and 1 as \textit{Actinobacteria} (Figure~\ref{fig:l2fc},
Figure~\ref{fig:trees}). If the responder OTUs are ranked by descending,
moderated proportion mean labeled:control ratios, the top 10 ratios (i.e. the
10 OTUs that were most enriched in the labeled gradients considering only heavy
fractions) are either \textit{Firmicutes} (6 OTUs) or \textit{Proteobacteria}
(4 OTUs) (Figure~\ref{fig:scatter_heavy}). Centroid sequences of
strongly responding \textit{Proteobacteria} OTUs all share high sequence identity
($>$98.48\%, Table~\ref{tab:LTP_blast}) with cultivars from genera
known to possess diazotrophs including \textit{Klebsiella}, \textit{Shigella},
\textit{Acinetobacter}, and \textit{Ideonella}. None of the \textit{Firmicutes}
OTUs in the top 10 responders share greater than 97\% sequence identity with
sequences in the LTP database (release 115) (see Table~\ref{tab:LTP_blast}).
OTUs that passed the sparsity threshold but were not classified as
$^{15}$N-responsive were subsequently tested against the null hypothesis that
the OTU proportion mean ratio was above the selected threshold. Rejecting the
second null would indicate an OTU did \textit{not} incorporate $^{15}$N into
biomass. There were 58 and 70 ``non-responders'' at days 2 and 4,
respectively. OTUs that did not pass sparsity or could not be classified as
either a responder or non-responder are simply ambiguous with respect to
$^{15}$N labelling.
\subsection{$^{15}$N-responsive OTUs in  environmental samples}
Five of the 6 \textit{Firmicutes} with the strongest response to
$^{15}$N-labelling (Table~X) belong in the \textit{Clostridiacea}. We only
observed one of these strongly responding \textit{Clostridiaceae} in the data
presented by \citet{Garcia_Pichel_2013}, "OTU.108" (closest BLAST hit in LTP
Release 115 -- \textit{Caloramotor proteoclasticus}, BLAST \%ID 96.94,
Accession X90488).  OTU.108 was found in two samples both characterized as
"light" crust. One other \textit{Clostridiaceae} OTU with a proportion mean
ratio (labeled:control) p-value less than 0.10 but outside the top 10
responders was found in the \citet{Garcia_Pichel_2013} data (a "light" crust
sample) (Figure~\ref{fig:trees}).  None of the strongly responding
\textit{Clostridiacea} were found in the sequences provided by
\citet{Steven_2013}. \textit{Clostridiaceae} $^{15}$N-responder OTU are not
closely related to cultivars.  (Table~\ref{tab:LTP_blast},
Figure~\ref{fig:clost_tree}).   One of the proteobacterial OTUs with the
strongest $^{15}$-N response (Table~X) was found in \citet{Garcia_Pichel_2013}
(closest BLAST hit in LTP Release 115, BLAST \%ID 100, Accession ZD3440,
\textit{Acinetobacter johnsonii}). None of the \textit{strongly responding}
\textit{Protebacteria} OTUs were found in the \citet{Steven_2013} sequences.
Responder OTUs were found in \citet{Steven_2013} samples 133 times. 83 were in
"below crust" samples, 50 in crust samples (see Figure~\ref{fig:trees}).  Two
$^{15}$N-responsive OTUs were found in an extensive number of environmental
samples (61 of 65 samples from the combined data sets of
\citet{Garcia_Pichel_2013} and \citet{Steven_2013}).  Both OTUs were annotated
as \textit{Acidobacteria} but shared little sequence identity to any cultivar
SSU rRNA gene sequences in the LTP (Release 115), with best LTP BLAST hits of
81.91 and 81.32\% identity. Additionally, the $^{15}$N-response for each OTU
was weak relative to other putative responders (\ref{fig:l2fc}. Of the
remaining 36 stable isotope responder OTUs, only 14 were observed in the
environmental data (Figure~\ref{fig:trees}, Figure~\ref{fig:rspndr_dist}).
\subsection{Comparing sequence collections at "study"-level}
We compared the sequences determined in this study to two previous surveys of
SSU rRNA amplicons from BSC communities: the \citet{Garcia_Pichel_2013}
and \citet{Steven_2013} study. There were 3,079 OTUs (209,354 total sequences
after quality control) in the DNA-SIP data, 3,203 OTUs (129,033 total sequences
after quality control) in the \citet{Garcia_Pichel_2013} study, and 2,481 OTUs
(129,358 total sequences after quality control) in the \citet{Steven_2013}
study. Of the 4,340 OTU centroids established for this study (including
sequences from \citet{Steven_2013} and \citet{Garcia_Pichel_2013}) 445 have
matches in the Living Tree Project (LTP) (a collection of 16S gene sequences
for all sequenced type strains \citep{Yarza_2008}) at greater or equal than
97\% sequence identity (LTP version 115). That is, 445 of 4,340 OTUs (\~10\%)
are closely related to cultivars. The DNA-SIP data set shares 56\% OTUs with
the \citet{Steven_2013} data and 46\% of OTUs with the
\citet{Garcia_Pichel_2013} data (where total OTUs are from the combined data
for each pairwise comparison).  The \citet{Steven_2013} and
\citet{Garcia_Pichel_2013} studies share 46\% of OTUs. \textit{Cyanobacteria}
and \textit{Proteobacteria} were the top two phylum-level sequence annotations
for all three studies of BSC. Only the DNA-SIP data had more
\textit{Proteobacteria} annotations than \textit{Cyanobacteria}.
\textit{Proteobacteria} represented the 29.8\% of sequence annotations in
DNA-SIP data as opposed to 17.8\% and 19.2\% for the \citet{Garcia_Pichel_2013}
and \citet{Steven_2013} data, respectively.  There is a stark contrast in the
total percentage of sequences annotated as \textit{Firmicutes} between the raw
environmental samples and the DNA-SIP data.  \textit{Firmicutes} represent only
0.21\% and 0.23\% of total phylum level sequence annotations in the
\citet{Steven_2013} and \citet{Garcia_Pichel_2013} studies, respectively
(Figure~\ref{fig:study_phy_dist}). In the DNA-SIP sequence collection
\textit{Firmicutes} make up 19\% of phylum level sequence annotations. SIP
places focus upon organisms based on isotope incorporation and has the ability
to detect activity by low abundance members of the community.  DNA from OTUs
that incopororate $^{15}$N into their biomass moves towards the heavy end of
the CsCl gradient and therefore OTUs in ``labeled'' DNA are enriched in the
full data pool relative to bulk DNA.  Phylum-level taxonomic annotations of
$^{15}$N-responsive OTUs (i.e. \textit{Firmicutes} and
\textit{Proteobacteria}) are enriched in the DNA-SIP data relative to
environmental data (Figure~\ref{fig:study_phy_dist}). Also in sharp contrast
for the DNA-SIP versus environmental data is the number of putative
heterocystous \textit{Cyanobacteria} sequences. Only 0.29\% of
\textit{Cyanobacteria} sequences in the DNA-SIP data are annotated as belonging
to "Subsection IV" which is the heterocystous order of \textit{Cyanobacteria}
in the Silva taxonomic nomenclature \citep{17947321}. In the
\citet{Steven_2013} and \citet{Garcia_Pichel_2013} studies 15\% and 23\%,
respectively, of \textit{Cyanobacteria} sequences are annotated as belonging to
"Subsection IV".  

