\section{Results}

\subsection{Comparison of sequence collections at "study"-level}

\subsubsection{Comparisons of OTU content}
Of the 4340 OTU centroids established for this study (including sequences from \citet{Steven_2013} and \citep{Garcia_Pichel_2013}) 445 and 870 have matches in the Living Tree Project (LTP) (a collection of 16S gene sequences for all sequenced type strains \citep{Yarza_2008}) at greater or equal than 97\% and 95\% sequence identity, respectively (LTP version 115). Similar numbers of total OTUs were found in each data set explored in this study (i.e. the DNA-SIP data presented here, the data presented by \citet{Steven_2013} and by \citet{Garcia_Pichel_2013}). Specifically, there were 3079 OTUs (209,354 total sequences after quality control) in the DNA-SIP data, 3203 OTUs (129,033 total sequences after quality control) in the \citet{Garcia_Pichel_2013} study, and 2481 OTUs (129,358 total sequences after quality control) in the \citet{Steven_2013} study. The DNA-SIP data set shares more OTUs with the \citet{Steven_2013} (56\% of total OTUs found in either of the two data sets) than it does with the \citet{Garcia_Pichel_2013} data (46\% of total OTUs between both data sets). The \citet{Steven_2013} and \citet{Garcia_Pichel_2013} only share 46\% of OTUs.

\subsubsection{Comparisons of Taxonomic Content} 
\textit{Cyanobacteria} and \textit{Proteobacteria} were the top two phylum-level sequence annotations for all three studies but only the DNA-SIP data had more \textit{Proteobacteria} annotations than \textit{Cyanobacteria}. \textit{Proteobacteria} represented the 29.8\% of sequence annotations in DNA-SIP data as opposed to 17.8\% and 19.2\% for the \citet{Garcia_Pichel_2013} and \citet{Steven_2013} data, respectively. Figure~\ref{fig:study_phy_dist} shows the distribution of phylum-level sequence annotations for each study in the nine most abundant  phyla across all studies, as determined by raw sequence counts. There is a stark contrast in the total percentage of sequences annotated as \textit{Firmicutes} between the raw environmental samples and the DNA-SIP data. \textit{Firmicutes} represent only 0.21\% and 0.23\% of total phylum level sequence annotations in the \citet{Steven_2013} and \citet{Garcia_Pichel_2013} studies, respectively. In the DNA-SIP sequence collection \textit{Firmicutes} make up 19\% of phylum level sequence annotations. Also in sharp contrast for the DNA-SIP versus environmental data is the number of putative heterocystous \textit{Cyanobacteria} sequences. Only 0.29\% of \textit{Cyanobacteria} sequences in the DNA-SIP data are annotated as belonging to "Subsection IV" which is the heterocystous order of \textit{Cyanobacteria} in the Silva taxonomic nomenclature \citep{17947321}. In the \citet{Steven_2013} and \citet{Garcia_Pichel_2013} studies 15\% and 23\%, respectively, of \textit{Cyanobacteria} sequences are annotated as belonging to "Subsection IV".  

\subsection{Ordination of CsCl gradient fraction SSU rRNA libraries}
Ordination of Bray-Curtis \citep{Bray_1957} distances between CsCl gradient
fraction sequence libraries with principal coordinates analysis shows the
labeled gradient fraction libraries diverge from control in the heavy fractions
(Figure~\ref{fig:ordination}). When the labeled and control gradient fractions
are paired such that each pair contains a control fraction and labeled fraction
from the same incubation day with a density difference below 0.003 g/mL, the
Bray-Curtis distance between the fraction pair is postively correlated to the
density of the labeled
fraction (p-value: 0.00052, r$^{2}$: 0.3315) (inset
Figure~\ref{fig:ordination}). Additionally, the label/control category for
heavy fractions is statistically significant by the Adonis test (p-value:
0.001, r$^{2}$: 0.136) \citep{Anderson_2001}. The first principal axis appears
to be correlated with fraction density (Figure~\ref{fig:ordination}) (Adonis test p-value for density with
all CsCl fraction libraries: 0.001, r$^{2}$ 0.117).

\subsection{Identities of possible $^{15}$N incorporators}
The OTUs that have enriched proportion means in labeled gradient heavy fractions 
versus control gradient heavy fractions are those that have incorporated to the
stable isotope tracer into their DNA which would indicate diazotropy in this
experiment. We found 38 responders total using a false discovery rate threshold
for multiple comparison
adjusted p-values of 10\%. Of these 38, 26 are annotated as
\textit{Firmicutes}, 9 as \textit{Proteobacteria}, 2 as \textit{Acidobacteria}
and 1 as \textit{Actinobacteria} (The inset of Figure~\ref{fig:l2fc} summarizes
the Family level taxanomic profile of stable isotope responders).
Figure~\ref{fig:l2fc} summarizes the ratio of proportion means for each OTU
where means are calculated from proportions in heavy fractions within labeled
or controlled gradients and the ratio is labeled over control (see methods). If
the OTUs are ranked by descending, moderated proportion mean labeled:control
ratios, the top 10 ratios (i.e. the 10 OTUs that were most enriched in the
labeled gradients in heavy fractions) are either \textit{Firmicutes} (6 OTUS)
or \textit{Proteobacteria} (4 OTUs). Figure~\ref{fig:scatter_heavy} shows the 
relative abundance values for the top 10 OTUs in heavy fractions of labeled and 
control gradients. Table~\ref{tab:LTP_blast} summarizes the results from BLAST
searching the centroid sequences for these top 10 OTUs against the LTP (version
115). The \textit{Proteobacteria} OTU centroid seqeunces for the top 10
responders all share high identity ($>$98.48\% identity, Table~\ref{tab:LTP_blast}) with
cultivars from genera known to possess diazotrophs including
\textit{Klebsiella}, \textit{Shigella}, \textit{Acinetobacter}, and
\textit{Ideonella}. None of the \textit{Firmicutes} OTUs in the top 10
responders share greater than 97\% sequence identity with sequences in the LTP
(relase 115) (see Table~\ref{tab:LTP_blast}).

\subsection{Distribution of BSC Diazotrophs in Environmental Samples}
\subsubsection{Non-Cyanobacterial Taxa}
\paragraph{\textit{Clostridiacea}}
Five of the 6 \textit{Firmicutes} in the top 10 responder OTUs (above) belong
in the \textit{Clostridiacea}. We only observed one of these strongly
responding
\textit{Clostrideacea} in the data presented by \citet{Garcia_Pichel_2013},
"OTU.108" (closest BLAST hit in LTP Relase 115 -- \textit{Caloramotor
proteoclasticus}, BLAST \%ID 96.94, Accession X90488). OTU.108 was found in two
samples both characterized as "light" crust. One other \textit{Clostrideacea}
OTU with a proportion mean ratio (labeled:control) p-value less than 0.10 but
outside the top 10 responders was found in the \citet{Garcia_Pichel_2013} data
and also in a "light" crust sample. None of the strongly responding
\textit{Clostridiacea} were found in the sequences provided by
\citet{Steven_2013}.

Figure~\ref{fig:clost_tree} depicts the phylogenetic breadth of
\textit{Clostridiaceae} N responder OTUs from this experiment. The phylogenetic
tree was constructed from nearly full-length reference sequences, and edge
width demonstrates the placements of short OTU centroid sequences in the
backbone tree (see methods for description of placement algorithm and selection
criteria for reference sequences). As shown, \textit{Clostridiaceae}
N-responder OTU centroid 16S sequences are generally more closely related to
environmental than cultivar 16S gene sequences.   

\paragraph{\textit{Gammaproteobacteria}} Only "OTU.342" (closest BLAST hit in
LTP Release 115, BLAST \%ID 100, Accession ZD3440, \textit{Acinetobacter
johnsonii}) of the \textit{Proteobacteria} OTUs in the top 10 most strongly
responding OTUs was found in the \citet{Garcia_Pichel_2013} sequences. None of
the strongly responding \textit{Protebacteria} OTUs were found in the
\citet{Steven_2013} sequences. There were 133 responder OTU-sample occurrences
(SIP responding OTU was found in a sample library) in the \citet{Steven_2013}
data. 83 were in "below crust" samples, 50 in BSC samples.

\paragraph{Other taxa} 
Two potentially diazotroph OTUs were found in an extensive number of
environmental samples (61 of 65 samples from the combined data sets of
\citet{Garcia_Pichel_2013} and \citet{Steven_2013}). Both OTUs were annotated
as \textit{Acidobacteria} but shared little sequence identity to any cultivar
SSU rRNA gene sequences in the LTP (Release 115), with best LTP BLAST hits of
81.91 and 81.32 \% identity. Additionally, the evidence for N incorporation for
each OTU was weak relative to other putative responders (adjusted p-values of
0.090 and 0.096). Of the remaining 36 stable isotope responder OTUs, only 14
were observed in the environmental data. Figure~\ref{fig:rspndr_dist}
summarizes the OTU-sample occurrences in both the \citet{Steven_2013} and the
\citet{Garcia_Pichel_2013} data with occurrences distributed into the most
relevant sample classes of each respective study.

\subsubsection{Heterocystous \textit{Cyanobacteria}} At least one of the six
OTUs defined by sequences recovered by \citet{Yeager} (see
Table~\ref{table:yeager_2006}) was found in 21 of the 23 sites surveyed by
\citet{Garcia_Pichel_2013}. Counts of samples with \citet{Yeager} sequence defined heterocystous cyanobacteria OTUs are summarize in Table~\ref{table:het_dist}. The opposite BSC relative abundance relationships
of \textit{Microcoleus Vaginatus} and \textit{M. Strenstrupii} with site mean
annual temperature was a major finding by \citet{Garcia_Pichel_2013}.
\citet{Garcia_Pichel_2013} did not report the relationship of diaztrophic
cyanobacteria with temperature although a comment by \citet{Belnap28062013}
briefly discusses a qualitative positive relationship of \textit{Scytonema}
with temperature in the \citet{Garcia_Pichel_2013} data. In agreement with the
\citet{Belnap28062013} interpretation we found a positive relationship of
\textit{Scytonema hyalinum} FGP-7A and DC-A OTU relative abundance with mean
annual temperature (p-values 3.332x10\textsuperscript{-03} and
3.173x10\textsuperscript{-04}, respectively) (Figure~\ref{fig:het_temp}). We
also found \textit{Nostoc commune} MCT-1 and MFG-1 OTU relative abundance was
inversely related to mean annual temperature (p-values
1.307x10\textsuperscript{-02} and 1.577x10\textsuperscript{-06}, respectively)
(Figure~\ref{fig:het_temp}). 

\begin{table}

\textbf{
    \refstepcounter{table}
    \label{table:het_dist}
    Table \arabic{table}.
}
    
    {Counts of heterocystous cyanobacterial OTU occurrences in \citet{Garcia_Pichel_2013} samples (n = 23) and \citet{Steven_2013} samples (n = 42)}

{\centering
\begin{tabular}{lrr}
  \toprule
 Isolate & \citet{Garcia_Pichel_2013} & \citet{Steven_2013} \\ 
  \midrule
Calothrix MCC-3A &   1 &   6 \\ \midrule
Nostoc commune MCT-1 &  16 &  23 \\ \midrule
Nostoc commune MFG-1 &  12 &  23 \\ \midrule
Scytonema hyalinum DC-A &  17 &  30 \\ \midrule
Scytonema hyalinum FGP-7A &  18 &  27 \\ \midrule
Spirirestis rafaelensis LQ-10 &  16 &  30 \\
   \bottomrule
\end{tabular}}{}
\end{table}

At least one OTU defined by selected 16S rRNA gene sequences presented by
\citet{Yeager} (Table~\ref{table:yeager_2006}) was found in all but 7 of 42
samples surveyed by \citet{Steven_2013} and all of these 7 samples that lacked
the \citet{Yeager} OTUs were "below crust" samples. Table~\ref{table:het_dist} summarizes the
counts \citet{Steven_2013} samples with \citet{Yeager} sequence defined OTUs. As expected all of the six OTUs defined by \citet{Yeager} sequences
were more abundant in the crust samples than below crust samples
(Figure~\ref{fig:het_steven}) (maximum p-value for any OTU: 1.96x10$^{-4}$).

\subsection{Richness estimates}
Figure~\ref{fig:rarefaction} (inset) summarizes the fraction of observed OTUs
over total OTUs as estimated by CatchAll for each sample 16S library.
Rarefaction curves for each sample are shown in Figure~\ref{fig:rarefaction}.
Qualitatively, rarefaction curves show below crust samples to be more rich than
BSC samples in the \citet{Steven_2013} data.
