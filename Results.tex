\section{Results}

\subsection{Comparison of sequence collections at "study"-level}
Of the 4340 OTU centroids established for this study (including sequences from \citet{Steven_2013} and \cite{Garcia_Pichel_2013}) 445 and 870 have matches in the Living Tree Project (LTP) a collection of 16S gene sequences for all sequenced type strains \cite{Yarza_2008} (LTP version 115).

Similar numbers of total OTUs were found in each data set explored in this study (i.e. the DNA-SIP data presented here, the data presented by \citet{Steven_2013} and by \citet{Garcia_Pichel_2013}). Specifically, there were 3079 OTUs (209,354 total sequences after quality control) in the DNA-SIP data, 3203 OTUs (129,033 total sequences after quality control) in the \citet{Garcia_Pichel_2013} study, and 2481 OTUs (129,358 total sequences after quality control) in the \citet{Steven_2013} study.

There is a stark contrast in the total percentage of sequences annotated as \textit{Firmicutes} between the raw environmental samples and the DNA-SIP data. \textit{Firmicutes} represent only 0.21\% and 0.23\% of total phylum level sequence annotations in the \citet{Steven_2013} and \citet{Garcia_Pichel_2013} study, respectively. In the DNA-SIP sequcuence collection, \textit{Firmicutes} make up 19\% of phylum level sequence annotations. \textit{Cyanobacteria} and \textit{Proteobacteria} were the top two phylum annotations for all three studies but only the DNA-SIP data had more \textit{Proteobacteria} annotations than \textit{Cyanobacteria}.

The DNA-SIP data set shares more OTUs with the \citet{Steven_2013} (56\% of total OTUs found in either of the two data sets) than it does with the \citet{Garcia_Pichel_2013} data (46\% of total OTUs between both data sets). The \cite{Steven_2013} and \cite{Garcia_Pichel_2013} share 46\% of total OTUs between the two sequence collections.

\subsection{Ordination of CsCl gradient fraction SSU rRNA libraries}
Ordination of Bray-Curtis \cite{Bray_1957} distances between CsCl gradient fraction sequence libraries with principal coordinates analysis shows the labelled gradient fraction libraries diverge from control in the heavy fractions (Figure X). When the labelled and control gradient fractions are paired such that each pair contains a control fraction and labelled fraction with a density difference below XXX g/mL, the Bray-Curtis distance between the fraction pair is proportional to the density of the labelled fraction (p-value 0.00052, r$^{2}$ 0.3315) (Figure X). Additionally, the label versus control for category for heavy fractions is statistically significant by the Adonis test (p-value 0.001, r$^{2}$ 0.136) \cite{Anderson_2001}. The first principal axes appears to be correlated with fraction density. (Adonis p-value for density all fraction 0.001, r$^{2}$ 0.117)

\subsection{Identities of possible $^{15}$N incorporators}
The OTUs that have proportion means in heavy fractions enriched in labelled gradients versus control are those that have responded to the stable isotope tracer which would indicate diazotropy in this experiment. We found 37 responders total at a false discovery rate of 10\%. Of these 37, 27 are annotated as \textit{Firmicutes}, 8 as \textit{Proteobacteria} and 1 each of \textit{Acidobacteria} and \textit{Actinobacteria}. Figure X summarizes the ratio of proportion means for each OTU where means are calculated from proportions in heavy fractions within labelled or controlled gradients and the ratio is labelled over control (see Methods SECTION). If the OTUs are ranked by desending, moderated proportion mean labelled:conrol ratios, the top 10 ratios (i.e. the 10 OTUs that were most enriched in the labelled gradients in heavy fractions) are either \textit{Firmicutes} (6 OTUS) or \text{Proteobacteria} (4 OTUs). Table X summarizes the results from BLAST searching the centroid sequences for these top 10 OTUs against the LTP (version 115).  
taxonomic summary

results for heterocystous cyanos

l2fc figure

LTP BLAST table

\subsubsection{Diversity of Clostridium $^{15}$N incorporators}

Clostridium tree

Placement in clades without cultured reps.

Summarize BLAST results (LTP and SSURefNR)

\subsubsection{Diversity of Gamma-proteobacterial $^{15}$N incorporators}

Taxonomic summary

Discuss gamma-proteos

Diazotrophs from DNA-SIP in "raw" datasets

Heterocystous cyanobacteria in "raw" datasets



