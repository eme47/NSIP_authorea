\section{Abstract}
Biological soil crusts (BSC) cover a vast global area and are key components of ecosystem productivity in arid soils. In particular, BSC contribute significantly to the nitrogen (N) budget in arid ecosystems via N-fixation. Although BSC N-fixation is largely attributed to heterocystous cyanobacteria \citep{Yeager, 14766579, Yeager_2012}, DNA stable isotope probing with \textsuperscript{15}N\textsubscript{2} revealed primarily \textit{Clostridiaceae} and \textit{Proteobacteria} incorporated N in mesocosm incubations with light, poorly developed BSC samples. Non-heterocystous BSC diazotrophs are low abundance members of BSC. The maximum relative abundance of putative \textit{Clostridiaceae} and \textit{Proteobacteria} diazotrophs in any SSU rRNA libraries presented by \citet{Garcia_Pichel_2013} or \citet{Steven_2013} was 0.00225\% and 0.00127\%, respectively. Heterocystous cyanobacteria relative abundance is correlated with mean annual temperature for \textit{Nostoc commune} MCT-1 and MFG-1, and \textit{Scytonema hyalinum} FGP-7A and DC-A (p-values 1.307x10\textsuperscript{-02}, 1.577x10\textsuperscript{-06} and 3.332x10\textsuperscript{-03}, 3.173x10\textsuperscript{-04}, respectively) but the direction of the correlation is different for \textit{Nostoc} (decreasing with temperature) and \textit{Scytonema} (increasing with temperature) types. Non-cyanobacterial diazotrophs have not been sampled sufficiently yet in existing BSC SSU rRNA sequence collections to diagnose their temperature relationships or geographic scope. Identifying the full BSC diazotroph diversity is an crucial step towards predicting how climate change and disturbance will and do affect BSC N-fixation.
