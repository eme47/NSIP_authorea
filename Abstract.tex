\section{Abstract}
Biological soil crusts (BSC) cover a vast global area and are key components of ecosystem productivity in arid soils. In particular, BSC contribute significantly to the nitrogen (N) budget in arid ecosystems. Although BSC N-fixation is largely attributed to heterocystous cyanobacteria \cite{Yeager, 14766579, Yeager_2012}, DNA stable isotope probing with \textsuperscript{15}N\textsubscript{2} revealed primarily \textit{Clostridiaceae} and \textit{Proteobacteria} incorporated N in mesocosm incubations with light, poorly developed BSC samples. Non-heterocystous BSC diazotrophs are low abundance members of BSC and were only found in XX\% and XX\% of SSU rRNA libraries presented by \citet{Garcia_Pichel_2013} and \citet{Steven_2013}, respectively. Heterocystous cyanobacteria relative abundance is correlated with mean annual temperature for \textit{Nostoc commune} MCT-1 and MFG-1 and \textit{Scytonema hyalinum} FGP-7A (p-values XXXX, XXXX and XXX, respectively). Non-cyanobacterial diazotrophs have not been sampled sufficiently yet in existing SSU rRNA sequence collections to diagnose their temperature relationships. Identifying the full BSC diazotroph diversity is an essential step towards predicting how climate change and disturbance will and do affect BSC N-fixation.