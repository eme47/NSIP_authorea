\section{Abstract}
Biological soil crusts (BSC) cover a vast global area and are key
components of ecosystem productivity in arid soils. In particular, BSC
contribute significantly to the nitrogen (N) budget
via N$_{2}$-fixation.  N$_{2}$-fixation in mature crusts is largely
attributed to heterocystous cyanobacteria, however, early successional
crusts possess few N-fixing cyanobacteria and this
suggests that microorganisms other than cyanobacteria mediate
N$_{2}$-fixation during the early stages of BSC development. DNA stable
isotope probing (DNA-SIP) with $^{15}$N$_{2}$ revealed that
\textit{Clostridiaceae} and \textit{Proteobacteria} are the most common
microorganisms to assimilate $^{15}$N in early successional 
crusts.  The maximum $^{15}$N$_{2}$-assimilating \textit{Clostridiaceae} and
\textit{Proteobacteria} Operational Taxonomic Unit (OTU, all sequences at 
least 97\% sequence identity to OTU seed) relative abundance in environmental
BSC SSU rRNA gene sequence surveys was 0.00225\% and 0.00127\% in any single
sample, respectively. Their low abundance may explain why these heterotrophic
diazotrophs have not previously been characterized in BSC. Diazotrophs play a
critical role in BSC formation and characterization of these organisms
represents a crucial step towards understanding how antropogenic change will
effect the formation and ecological function of BSC in arid ecosystems.

%Heterocystous cyanobacteria relative abundance is correlated with mean annual
%temperature for \textit{Nostoc commune} MCT-1 and MFG-1, and \textit{Scytonema
%hyalinum} FGP-7A and DC-A (p-values 1.307x10$^{-02}$, 1.577x10$^{-06}$ and
%3.332x10$^{-03}$, 3.173x10$^{-04}$, respectively). However, the direction of
%the correlation is different for \textit{Nostoc} (decreasing with temperature)
%and \textit{Scytonema} (increasing with temperature) types.  

