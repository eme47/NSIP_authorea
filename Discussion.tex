\section{Discussion}
\subsection{Study-level differences}
SIP places focus upon organisms based on isotope incorporation and has the
ability to detect activity by low abundance members of the community.
DNA from OTUs that incopororate $^{15}$N into their biomass moves towards the
heavy end of the CsCl gradient and therefore OTUs in ``labeled'' DNA are
enriched in the full data pool relative to bulk DNA. Phylum-level taxonomic
annotations such as \textit{Firmicutes} and \textit{Proteobacteria} of
$^{15}$N-responsive OTUs are enriched in the DNA-SIP data relative to
environmental data (Figure X).

\subsection{Ordination of CsCl gradient fraction 16S rRNA gene sequence collections} 
The ordination of Bray-Curtis distances between CsCl gradient fraction 16S
sequence collections show that control fractions differ from labeled fractions
in the "heavy" range of the CsCl gradients (Figure~\ref{fig:ordination}). If
each control fraction is paired to the labeled fraction from the same
incubation day for which it is closest in density, there is a positive and
statistically significant correlation between Bray-Curtis distances within
fraction pairs and density of the pair (see inset Figure~\ref{fig:ordination}).
Therefore, the "heavy" end of the control and labeled gradients differ and the
OTUs enriched in the labeled fractions (relative to control) would have
incorporated $^{15}$N into their DNA during the incubation timeframe. If the
incubation timeframe is appropriate, the $^{15}$N-incorporators would most
likely have incorporated the $^{15}$N from atmospheric $^{15}$N$_{2}$.     

\subsection{BSC diazotrophs identified in the study} BSC N-fixation has long
been attributed to heterocystous \textit{Cyanobacteria} and molecular microbial
ecology surveys of BSC \textit{nifH} gene content have been consistent with
this hypothesis finding cyanobacterial \textit{nifH} types to be numerically
dominant in \textit{nifH} gene libraries \citep{Yeager,14766579,Yeager_2012}.
Even poorly developed BSC samples have yielded predominantly cyanobacterial
\textit{nifH} genes \citep{14766579}. And,
"sub-biocrust" samples have yielded \textit{entirely} heterocystous
cyanobacterial \textit{nifH} genes \citep{Yeager_2012}.
It is possible, however, that PCR-driven molecular surveys of \textit{nifH}
gene content have been biased against non-heterocystous \textit{Cyanobacteria}.
In general the nifH PCR primers used by \citet{Yeager,14766579,Yeager_2012}
(19F and nifH3) for the first round of nested PCR have broad specificity and
display at least 86\% \textit{in silico} coverage for \textit{Proteobacteria,
Cyanobacteria} and ``Cluster III'' \textit{nifH} reference sequences
\citep{Gaby_2012}. In the second round of the nested PCR protocol
\citep{Yeager,14766579,Yeager_2012}, primer nifH11 is` slightly biased against
``Cluster III'' (50\% coverage) but biased in favor of \textit{Proteobacteria}
(79\% \textit{in silico} coverage against 67\% for \textit{Cyanobacteria}) and
nifH22 matches \textit{Proteobacteria}, \textit{Cyanobacteria} and ``Cluster
III'' reference sequences poorly (16\%, 23\% and 21\% \textit{in silico}
coverage, respectively) \citep{Gaby_2012}.  Unfortunately, it is difficult to
assess or quantify this bias (in either direction) without knowing the
\textit{nifH} gene content \textit{de novo}. Another potential bias in favor of
\textit{Cyanobacteria} in BSC \textit{nifH} gene libraries is heterocysts (the
specialized N-fixing cells along the trichome of filamentous heterocystous
\textit{Cyanobacteria} such as \textit{Nostoc} and \textit{Scytonema}) may be
overrepresented with respect to non-cyanobacterial diazotrophs because
heterocysts make up a fraction cells along a trichome and even non-heterocyst
cells in a trichome will possess the \textit{nifH} gene. Polyploidy could
further exacerbate this bias, as many \textit{Cyanobacteria} are estimated to
have multiple genome copies per cell \citep{Griese_2011}. Moreover, it should
also be noted that \textit{nifH} gene content is not directly extrapolable to
the taxonomic relative abundances of nitrogenase proteins. Regardless, our
results suggest that BSC N-fixation may include a significant
non-cyanobacterial component that requires further assessment across a more
comprehensive sampling of BSC types.

We did not observe evidence for N-fixation by heterocystous
\textit{Cyanobacteria} in the "light" crust samples used in this study. One
possible explanation for our results is that the "light", still developing
BSC samples used in this study possessed too few heterocystous
\textit{Cyanobacteria} to statistically evaluate their $^{15}$N-incorporation.
Indeed, only 0.29\% of sequences from this study's DNA-SIP 16S rRNA gene
sequence libraries were from heterocystous \textit{Cyanobacteria} (see results)
as opposed to 15\% and 23\% of total sequences in the \citet{Steven_2013} and
\citet{Garcia_Pichel_2013} data, respectively. It is difficult to compare
relative abundance values from CsCl gradient fractions against environmental
libraries, but, a three order of magnitude difference between the
environmental libraries and the CsCl gradient fractions is stark.
Nonetheless, we would still expect even low abundance diazotrophs to show
evidence for $^{15}$N-incorporation, provided sequence counts were not too
sparse in heavy fractions. The OTUs defined by selected heterocystous
\textit{Cyanobacteria} sequences presented in \citet{Yeager}, however, all
fall below the sparsity threshold used in our analysis (see methods,
Figure~\ref{fig:het_sparsity}). Given the sparsity of heterocystous
\textit{Cyanobacteria} sequences in the DNA-SIP data set, it is not possible
to assess whether heterocystous \textit{Cyanobacteria} incorporated $^{15}$N
during the incubation. It should be noted that "light" and in particular
"sub-biocrust" samples possess much less heterocystous \textit{Cyanobacteria}
in general (Figure~\ref{fig:het_steven}) so the samples used in this study are
not necessarily unrepresentative of typical poorly developed BSC simply because
they are lacking heterocystous \textit{Cyanobacteria}. 

The OTUs that did appear to incorporate $^{15}$N during the incubation were
predominantly \textit{Proteobacteria} and \textit{Firmicutes}. The
\textit{Proteobacteria} OTUs for which $^{15}$N-incorporation signal was
strongest all shared high sequence identity (\textgreater=98.48\% sequence
identity) with 16S sequences from cultivars in genera with known diazotrophs
(Table~\ref{tab:LTP_blast}). The \textit{Firmicutes} that displayed signal
for $^{15}$N-incorporation (predominantly \textit{Clostridiaceae}) were not
closely related to any cultivars (Table~\ref{tab:LTP_blast},
Figure~\ref{fig:clost_tree}). These BSC \textit{Clostrodiaceae} diazotrophs
represent a gap in culture collections. As culture-based ecophysiological
studies have proven useful towards explaining ecological phenomena in BSC 16S
rRNA gene sequence libraries \citep{Garcia_Pichel_2013}, it would seem that
these putative \textit{Clostridiaceae} diazotrophs would be prime candidates
for targeted culturing efforts. Assessing the physiological response of these
diazotrophic \textit{Clostridiaceae} to temperature would be useful for
predicting how climate change will affect the BSC nitrogen budget. 

\textbf{This would be place to mention that C. pasteuranium was firs known diazotroph,
isolated from soil by Wintogradsky.

Also the place to mention cyanobacterial compatible solutes and their dynamics
in relation to wetting of dry soil. This relates to spore formes with a
boom-bust ecology.}

Although too undersampled in the environmental data sets to reach statistical
conclusions, $^{15}$N-responsive OTUs were found more often in below
crust samples (as opposed to BSC samples) in the \citet{Steven_2013} data and
in "light" BSC samples in the \citet{Garcia_Pichel_2013} data
(Figure~\ref{fig:rspndr_dist}). This result generates some hypotheses that
are counter to prior discussions regarding BSC diazotroph temporal dynamics
(keeping in mind this phenomenon has not been evaluated statistically).
Specifically, the transition of BSC from a light colored, developing crust to
a dark, mature crust may not mark the \textit{emergence} of diazotrophs in
BSC but rather the \textit{transition} of the diazotroph community from
heterotroph dominance to cyanobacterial.  Additionally, the soil beneath BSC
may contribute significantly to the N budget in arid ecosystems.


\subsection{Sequencing depth}
Rarefaction curves of all samples from \citet{Steven_2013} and
\citet{Garcia_Pichel_2013} are still sharply increasing especially for "below
crust" samples (Figure~\ref{fig:rarefaction}). Parametric richness estimates of
BSC diversity indicate the \citet{Steven_2013} and \citet{Garcia_Pichel_2013}
sequencing efforts recovered on average 40.5\% (sd. 9.99\%) and 45.5\% (sd.
11.6\%) of existing 16S OTUs from samples (inset Figure~\ref{fig:rarefaction}),
respectively. Further, the \citet{Steven_2013} and \citet{Garcia_Pichel_2013}
sequence collections only share 57.6\% of total OTUs found in at least one of
the studies. In fact, this study shares more OTUs with \citet{Steven_2013},
62.4\% of OTUs in the combined data, than the \citet{Steven_2013} study shares
with \citet{Garcia_Pichel_2013}.  Therefore, is not alarming that few of the
$^{15}$N-responsive OTUS were found by \citet{Garcia_Pichel_2013} and
\citet{Steven_2013}, it is important to point out that even next-generation
sequencing efforts of BSC 16S rRNA genes have only shallowly sampled the full
diversity of BSC microbes.  
%\subsection{Temperature influences on heterocystozus \textit{Cyanobacteria}
%relative abundance} Although few putative diazotrophs identified by DNA-SIP
%were found in the \citet{Garcia_Pichel_2013} and \citet{Steven_2013} data, we
%did observe statistically significant relationships between several
%heterocystous cyanobacterial OTUs with site mean annual temperature.
%Specifically, we found \textit{Nostoc commune} MCT-1 and MFG-1 relative
%abundances were negatively correlated with sample mean annual temperature.
%Additionally, it appears that the relative abundances of \textit{Scytonema
%hyalinum} FGP-7A and DC-A are positively correlated with mean annual
%temperature.
%
%\citet{Yeager_2012} found \textit{nifH} gene abundance peaks in early summer
%and falls in autumn. Although \citet{Yeager_2012} also experimentally
%increased the ambient temperature of several BSC samples over a long period
%(up to two years), changes in ambient temperature did not influence
%\textit{nifH} gene abundance as measured by qPCR. We are not able to confirm
%these results using the data from \citet{Garcia_Pichel_2013}, which is
%compositional in nature as opposed to absolute, but it does appear that
%the structure of heterocystous cyanobacterial diazotroph
%communities is correlated to mean annual temperature if not the absolute
%abundance of \textit{nifH} genes. 
\subsection{Conclusion}
Heterocystous \textit{Cyanobacteria} are key contributors to the BSC
N-budget, but, the putative diazotrophs elucidated in this study and in
\citet{Steppe_1996} in addition to the N-fixation rate data presented by
\citep{15643930} suggest there may be significant
non-cyanobacterial BSC diazotrophs specifically within the
\textit{Clostrideaceae} and \textit{Proteobacteria}. It seems clear that
heterocystous \textit{Cyanobacteria} increase in abundance with BSC age
\citep{14766579}. It is less clear if this transition marks the emergence of
diazotrophy versus a re-structuring of the BSC diazotroph community from one
dominated by \textit{Firmicutes} and \textit{Proteobacteria} to one
predominantly heterocystous \textit{Cyanobacteria}. DNA-SIP is a valuable tool in the
molecular microbial ecologist's toolbox for identifying members of microbial
community functional guilds \citep{17446886}. PCR-based surveys of diagnostic
marker genes and DNA-SIP are both used to connect microbial phylogenetic
types to microbial activities, but they occupy a non-overlapping set of
strengths and weaknesses.  Combined these tools can powerfully reveal
connections between ecosystem membership/structure and function. Here we
supplement previous surveys of BSC \textit{nifH} diversity, a diagnostic
marker PCR-driven approach, with $^{15}$N$_{2}$ DNA-SIP, and, while we do not
confirm previous results, we expand knowledge of BSC diazotroph diversity.
Predicting BSC N-fixation with respect to climate change, althered 
precipitation regimes and physical disturbance requires a careful accounting of
diazotrophs including non-cyanobacterial types. 
