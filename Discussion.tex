\section{Discussion}

\subsection{BSC Non-Cyanobacterial Diazotrophs}
BSC N-fixation has long been attributed to heterocystous cyanobacteria and molecular microbial ecology surveys of BSC \textit{nifH} gene content have been consistent with this hypothesis finding cyanobacterial \textit{nifH} types to be numerically dominant in \textit{nifH} gene libraries \cite{Yeager, 14766579, Yeager_2012}. It's possible, however, that PCR driven molecular surveys of \textit{nifH} gene content have been biased against non-heterocystous cyanobacteria (CITE GABY). Unfortunately, it is impossible to assess or quantify this bias (in either direction) without knowing the \textit{nifH} gene content \textit{de novo}. Perhaps non-PCR based molecular data such as metagenomic DNA sequence libraries will provide additional evidence with respect to the relative abundances of BSC \textit{nifH} gene types. Additionally, heterocysts (the specialized N-fixing cells along the trichome of filamentous heterocystous cyanobacteria such as \textit{Nostoc} and \textit{Scytonema}) may be overepresented with respect to non-heterocyst N-fixing cells in \textit{nifH} libraries because the heterocysts make up a fraction of the total cells along a trichome in and even the non-heterocyst cells in a trichome will possess the \textit{nifH} gene. It should also be noted that \textit{nifH} gene content is not a direct extrapolable to the relative abundances of nitrogenase proteins.

We did not observe evidence for N-fixation by heterocystous cyanobacteria in the "light" crust samples used in this study. One possible explanation for our results is that the the "light", still developing BSC samples used in this study possessed less heterocystous cyanobacteria than dark mature BSC as has been observed in previous comparisons of light and dark BSC \cite{14766579}. Indeed, only 0.29\% of seqeunces from this study's DNA-SIP 16S rRNA gene seqeunce libraries were from heterocystous cyanobacteria (see results) as opposed to 15\% and 23\% of total sequences in the \citet{Steven_2013} and \citet{Garcia_Pichel_2013} data, respectively. It's difficult to compare relative abundance values from CsCl gradient fractions against environmental libraries, but, a three order of magnitude difference between the environmental librares and the CsCl gradient fractions is stark. Nonetheless, we would still expect even low abundance diazotrophs to still show evidence for N-incorporation provided sequence counts were not too sparse in heavy fractions but the OTUs defined by selected heterocystous cyanobacteria sequences presented in \citet{Yeager} all fall below the sparsity threshold (Figure X). Given the sparsity of heterocystous cyanobacteria sequences in the DNA-SIP data set it is not possible to assess whether heterocystous cyanobacteria incorporated N during the incubation.

Regardless, the taxa OTUs that did appear to incoporate N during the incubation were predominantly \textit{Proteobacteria} and \textit{Firmicutes}. The \textit{Proteobacteria} OTUs for which signal for N-incorporation was strongest all shared high sequence identity (\textgreater=98.48\% sequence identity) with 16S sequences from cultivars in genera with known diazotrophs. The \textit{Firmicutes} (predominantly \textit{Clostridiaceae} that displayed signal for N-incorporation were not closely related to any cultivars (Table X, Figure X). There appears to be a gap in culture collections for these BSC diazotrophs. As ecophysiological studies have proven useful towards explaining ecological phenomena in BSC 16S rRNA gene sequence libraries \cite{Garcia_Pichel_2013}, it would seem that these putative \textit{Clostridiaceae} diazotrophs would be prime candidates for targeted culturing efforts. Assessing the physiological response of these \texit{Clostridiaceae} to temperature would be useful towards predicting how climate change will affect the BSC nitrogen budget.   

\cite{Yeager_2012} trends in het cyano w/ temp

Sequencing depth

Heterocyst abundance along trichome
DNA-SIP continuous result. "Boundary" in results that denotes responders is artificial. Eco-phyiological studies with cultivars to confirm results CITE THE BELNAP STUDY.

