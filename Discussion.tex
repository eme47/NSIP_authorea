\section{Discussion}

\subsection{BSC Non-Cyanobacterial Diazotrophs}
BSC N-fixation has long been attributed to heterocystous cyanobacteria and molecular microbial ecology surveys of BSC \textit{nifH} gene content have been consistent with this hypothesis finding cyanobacterial \textit{nifH} types to be numerically dominant in \textit{nifH} gene libraries \cite{Yeager, 14766579, Yeager_2012}. It's possible, however, that PCR driven molecular surveys of \textit{nifH} gene content have been biased against non-heterocystous cyanobacteria (CITE GABY). Unfortunately, it is impossible to assess or quantify this bias (in either direction) without knowing the \textit{nifH} gene content \textit{de novo}. Additionally, heterocysts (the specialized N-fixing cells along the trichome of filamentous heterocystous cyanobacteria such as \textit{Nostoc} and \textit{Scytonema}) may be overepresented with respect to non-heterocyst N-fixing cells in \textit{nifH} libraries because the heterocysts make up a fraction of the total cells along a trichome. Perhaps non-PCR based molecular data such as metagenomic DNA sequence libraries will provide additional evidence with respect to the relative abundances of BSC \textit{nifH} gene types. It should also be noted that \textit{nifH} gene content is not a direct extrapolable to the relative abundances of nitrogenase proteins.

We did not observe evidence for N-fixation by heterocystous cyanobacterial in the "light" crust samples used in this study. We see several reasons that may account for this...

\cite{Yeager_2012} trends in het cyano w/ temp

Sequencing depth

Heterocyst abundance along trichome
DNA-SIP continuous result. "Boundary" in results that denotes responders is artificial. Eco-phyiological studies with cultivars to confirm results CITE THE BELNAP STUDY.

