\section{Discussion}

\subsection{Ordination of CsCl gradient fraction 16S libraries}
pass

\subsection{BSC non-cyanobacterial diazotrophs}
BSC N-fixation has long been attributed to heterocystous cyanobacteria and molecular microbial ecology surveys of BSC \textit{nifH} gene content have been consistent with this hypothesis finding cyanobacterial \textit{nifH} types to be numerically dominant in \textit{nifH} gene libraries \cite{Yeager,14766579,Yeager_2012}. It is possible, however, that PCR-driven molecular surveys of \textit{nifH} gene content have been biased against non-heterocystous cyanobacteria (CITE GABY). Unfortunately, it is impossible to assess or quantify this bias (in either direction) without knowing the \textit{nifH} gene content \textit{de novo}. Perhaps non-PCR based molecular data such as metagenomic DNA sequence libraries will provide additional evidence with respect to the relative abundances of BSC \textit{nifH} gene types. Additionally, heterocysts (the specialized N-fixing cells along the trichome of filamentous heterocystous cyanobacteria such as \textit{Nostoc} and \textit{Scytonema}) may be overepresented with respect to non-heterocyst N-fixing cells in \textit{nifH} libraries because the heterocysts make up a fraction of the total cells along a trichome and even the non-heterocyst cells in a trichome will possess the \textit{nifH} gene. It should also be noted that \textit{nifH} gene content is not directly extrapolable to the taxonomic relative abundances of nitrogenase proteins.

We did not observe evidence for N-fixation by heterocystous cyanobacteria in the "light" crust samples used in this study. One possible explanation for our results is that the the "light", still developing BSC samples used in this study possessed less heterocystous cyanobacteria than dark mature BSC as has been observed in previous comparisons of light and dark BSC \cite{14766579}. Indeed, only 0.29\% of seqeunces from this study's DNA-SIP 16S rRNA gene sequence libraries were from heterocystous cyanobacteria (see results) as opposed to 15\% and 23\% of total sequences in the \citet{Steven_2013} and \citet{Garcia_Pichel_2013} data, respectively. It is difficult to compare relative abundance values from CsCl gradient fractions against environmental libraries, but, a three order of magnitude difference between the environmental librares and the CsCl gradient fractions is stark. Nonetheless, we would still expect even low abundance diazotrophs to show evidence for N-incorporation, provided sequence counts were not too sparse in heavy fractions. The OTUs defined by selected heterocystous cyanobacteria sequences presented in \citet{Yeager}, however, all fall below the sparsity threshold used in our analysis (see methods, Figure X). Given the sparsity of heterocystous cyanobacteria sequences in the DNA-SIP data set, it is not possible to assess whether heterocystous cyanobacteria incorporated N during the incubation.

The OTUs that did appear to incoporate N during the incubation were predominantly \textit{Proteobacteria} and \textit{Firmicutes}. The \textit{Proteobacteria} OTUs for which N-incorporation signal was strongest all shared high sequence identity (\textgreater=98.48\% sequence identity) with 16S sequences from cultivars in genera with known diazotrophs (Table X). The \textit{Firmicutes} that displayed signal for N-incorporation (predominantly \textit{Clostridiaceae}) were not closely related to any cultivars (Table X, Figure X). There appears to be a gap in culture collections for these BSC diazotrophs. As culture-based ecophysiological studies have proven useful towards explaining ecological phenomena in BSC 16S rRNA gene sequence libraries \cite{Garcia_Pichel_2013}, it would seem that these putative \textit{Clostridiaceae} diazotrophs would be prime candidates for targeted culturing efforts. Assessing the physiological response of these \textit{Clostridiaceae} to temperature would be useful towards predicting how climate change will affect the BSC nitrogen budget. 

Although somewhat undersampled in the environmental data sets to reach statistical conclusions, non-heterocystous diazotrophs were found more often in below crust samples (as opposed to actual BSC samples) from the \citet{Steven_2013} and in "light" BSC samples in the \citet{Garcia_Pichel_2013} data (Figure X). This result generates some hypotheses that are counter to prior conjecture regarding BSC diazotroph temporal dynamics (keeping in mind this phenomenon has not been evaluated statistically). Specifically, the transition of BSC from a light colored, developing crust to a dark, mature crust may not mark the emergence of diazorophs in BSC but rather the transition of the diazotroph community from heterotroph dominance to cyanobacterial. Additionally, the soil beneath BSC may contribute significantly to the N budget in arid ecosystems.

It is unclear why BSC \textit{nifH} gene surveys have overwhelmingly recovered heterocystous, cyanobacterial \textit{nifH} genes which would be in contrast to our results. Even poorly developed BSC samples have yielded predominantly cyanobacterial \textit(nifH) genes \cite{14766579}. And, "sub -biocrust" samples have yielded entirely heterocystous cyanobacterial \textit{nifH} genes \cite{Yeager_2012}. One explanation is that the samples from this study are simply different in diazotrophic community structure than those surveyed in \citet{Yeager} \citet{14766579} and \citet{Yeager_2012}. Indeed, it appears that the "light" crusts used here had a paucity of heterocystous cyanobacteria from the beginning (see above). Additionaly, cyanobacterial \textit{nifH} genes would be found in every heterocystous cyanobacterial cell, not just the heterocysts. Therefore, the contribution of heterocystous cyanobacteria to BSC N-fixation may be much less than the relative abundance of heterocystous cyanobacteria in \textit{nifH} gene libraries. Polyploidy could further exacerbate this bias as many cyanobacteria are estimated to have multiple genome copies per cell \cite{Griese_2011}. In any case, the DNA-SIP discovered diazotrophs for the "light", poorly developed BSC used in the study were not cyanobacterial but it is unknown if non-cyanobacterial diazotrophs would be identified by DNA-SIP with $^{15}$N using mature BSC samples. 

\subsection{Sequencing depth}
While it is somewhat alarming how few of the putative diazotrophs found in this study were also found by \citet{Garcia_Pichel_2013} and \citet{Steven_2013}, it is important to point out that even next-generation sequencing efforts of BSC 16S rRNA genes have only shallowly sampled the full diversity of BSC microbes. Rarefaction curves of all samples from \citet{Steven_2013} and \citet{Garcia_Pichel_2013} are still sharply increasing especially for "below crust" samples (Figure X). Parametric richness estimates of BSC diversity indicate the \citet{Steven_2013} and \citet{Garcia_Pichel_2013} sequencing efforts recovered on average 40.5\% (sd. 9.99\%) and 45.5\% (sd. 11.6\%) of existing 16S OTUs from samples (Figure X), respectively. Further, the \citet{Steven_2013} and \citet{Garcia_Pichel_2013} only share 57.6\% of total OTUs found in at least one of the studies. In fact, this study shares more OTUs with \citet{Steven_2013}, 62.4\% of total OTUs between both studies, than the \citet{Steven_2013} study shares with \citet{Garcia_Pichel_2013}. 

\subsection{Temperature influences on heterocystous cyanobacteria relative abundance}
Although few putative diazotrophs identitied by DNA-SIP were found in the \citet{Garcia_Pichel_2013} and \citet{Steven_2013} data, we did make some new observations regarding the relationship of several heterocystous cyanobacterial OTUs with site mean annual temperature. Specifically, we found \textit{Nostoc commune} MCT-1 and MFG-1 relative abundances were negatively correlated with sample mean annual temperature. Additionally, it appears that the relative abundances \textit{Scytonema hyalinum} FGP-7A  are positively correlated with mean annual temperature.

\citet{Yeager_2012} found \textit{nifH} gene abundance changes seasonally peaking in early summer and falling in autumn. Although \citet{Yeager_2013} also experimentally increased the ambient temperature of several BSC samples over a long period (up to two years), changes in ambient temperature did not influence \textit{nifH} gene abundance. We are not able to confirm these results using the data from \citet{Garcia_Pichel_2013} which is compositional in nature as opposed to absolute but it does appear that temperature affects the structure of heterocystous cyanobacterial diazotroph communities if not the absolute abundance of \textit{nifH} genes. 

\subsection{Analysis of next-generation-sequencing DNA-SIP data}
Although DNA-SIP is a powerful technique, analysis of DNA-SIP data is not without ambiguities. One limitation is the artificial boundary in the form of a selected adjusted p-value threshold (or false discovery rate) that marks which OTUs we consider to have incorporated N into their DNA during the incubation. In reality the metric we use to quantify the magnitude of an OTU's response to a stable isotope is continuous and there is not a non-artificial boundary between which OTUs appear to have "responded" and which OTUs have unknown response. For this reason, we have presented the OTUs that satisfy our "response" criteria, while focusing on the most strongly responding OTUs. As with any hypthesis-based statistical test, care should be taken when interpreting the significance of results where p-values are near the selected "significance" threshold for rejecting the null hypothesis.

Eco-phyiological studies with cultivars to confirm results CITE THE BELNAP STUDY.

