\section{Discussion}
\subsection{Study-level differences}
The most striking difference between the environmental datasets
\citep{Garcia_Pichel_2013, Steven_2013} and the DNA-SIP data is the difference
in the relative abundance of \textit{Firmicutes} sequence annotations
(Figure~\ref{fig:study_phy_dist}). The DNA-SIP data also has significantly more
\textit{Proteobacteria} sequence annotations (Figure~\ref{fig:study_phy_dist}).
The increased \textit{Firmicutes} and \textit{Proteobacteria} annotations are
consistent with the phylum-level taxonomies of the most strongly $^{15}$N
responding OTUs (see results). At the distal ends of a CsCl DNA-SIP gradient
there is little DNA, but, since we are working with compositional data and
gradient fraction libraries are not weighted by absolute DNA content, OTUs
found at the ends of CsCl gradients are inflated in overall abundance relative
to their abundance in the non-fractionated DNA. DNA from OTUs that incopororate
$^{15}$N into their biomass moves towards the heavy end of the CsCl gradient
and therefore OTUs in this ``labeled'' DNA are enriched in the full data pool
relative to environmental DNA. 

\subsection{Ordination of CsCl gradient fraction 16S libraries}
The ordination of Bray-Curtis distances between CsCl gradient fraction 16S
libraries for each day show that control fractions differ from labeled
fractions in the "heavy" range of the CsCl gradients
(Figure~\ref{fig:ordination}). If each control fraction is paired to the
labeled fraction from the same incubation day that it is closest in density to
and the Bray-Curtis distances for each pair are plotted against the density of
the labeled fraction, there is a positive and statistically significant
correlation between Bray-Curtis distance and density (see inset
Figure~\ref{fig:ordination}).  Therefore, the "heavy" end of the control and
labeled gradients differ and the
OTUs enriched in the labeled fractions would have incorporated $^{15}$N into their DNA
during the incubation timeframe. If the incubation timeframe is appropriate,
the $^{15}$N-incorporators would most likely incorporated the $^{15}$N from
atmospheric $^{15}$N$_{2}$.     

\subsection{BSC diazotrophs identified in the study}
BSC N-fixation has long been attributed to heterocystous cyanobacteria and
molecular microbial ecology surveys of BSC \textit{nifH} gene content have been
consistent with this hypothesis finding cyanobacterial \textit{nifH} types to
be numerically dominant in \textit{nifH} gene libraries
\citep{Yeager,14766579,Yeager_2012}. It is possible, however, that PCR-driven
molecular surveys of \textit{nifH} gene content have been biased against
non-heterocystous cyanobacteria \citep{Gaby_2012}. Unfortunately, it is impossible to
assess or quantify this bias (in either direction) without knowing the
\textit{nifH} gene content \textit{de novo}. Perhaps non-PCR based molecular
data such as metagenomic DNA sequence libraries will provide additional
evidence with respect to the relative abundances of BSC \textit{nifH} gene
types. Additionally, heterocysts (the specialized N-fixing cells along the
trichome of filamentous heterocystous cyanobacteria such as \textit{Nostoc} and
\textit{Scytonema}) may be overrepresented with respect to non-heterocyst
N-fixing cells in \textit{nifH} libraries because the heterocysts make up a
fraction of the total cells along a trichome and even the non-heterocyst cells
in a trichome will possess the \textit{nifH} gene. It should also be noted that
\textit{nifH} gene content is not directly extrapolable to the taxonomic
relative abundances of nitrogenase proteins.

We did not observe evidence for N-fixation by heterocystous cyanobacteria in
the "light" crust samples used in this study. One possible explanation for our
results is that the "light", still developing BSC samples used in this
study possessed less heterocystous cyanobacteria than dark mature BSC as has
been observed in previous comparisons of light and dark BSC \citep{14766579}.
Indeed, only 0.29\% of sequences from this study's DNA-SIP 16S rRNA gene
sequence libraries were from heterocystous cyanobacteria (see results) as
opposed to 15\% and 23\% of total sequences in the \citet{Steven_2013} and
\citet{Garcia_Pichel_2013} data, respectively. It is difficult to compare
relative abundance values from CsCl gradient fractions against environmental
libraries, but, a three order of magnitude difference between the environmental
libraries and the CsCl gradient fractions is stark. Nonetheless, we would still
expect even low abundance diazotrophs to show evidence for $^{15}$N-incorporation,
provided sequence counts were not too sparse in heavy fractions. The OTUs
defined by selected heterocystous cyanobacteria sequences presented in
\citet{Yeager}, however, all fall below the sparsity threshold used in our
analysis (see methods, Figure~\ref{fig:het_sparsity}). Given the sparsity of
heterocystous cyanobacteria sequences in the DNA-SIP data set, it is not
possible to assess whether heterocystous cyanobacteria incorporated $^{15}$N during
the incubation.

The OTUs that did appear to incorporate $^{15}$N during the incubation were
predominantly \textit{Proteobacteria} and \textit{Firmicutes}. The
\textit{Proteobacteria} OTUs for which $^{15}$N-incorporation signal was
strongest all shared high sequence identity (\textgreater=98.48\% sequence
identity) with 16S sequences from cultivars in genera with known diazotrophs
(Table~\ref{tab:LTP_blast}). The
\textit{Firmicutes} that displayed signal for $^{15}$N-incorporation (predominantly
\textit{Clostridiaceae}) were not closely related to any cultivars
(Table~\ref{tab:LTP_blast}, Figure~\ref{fig:clost_tree}). There appears to be a
gap in culture collections for these BSC diazotrophs. As culture-based
ecophysiological studies have proven useful towards explaining ecological
phenomena in BSC 16S rRNA gene sequence libraries \citep{Garcia_Pichel_2013},
it would seem that these putative \textit{Clostridiaceae} diazotrophs would be
prime candidates for targeted culturing efforts. Assessing the physiological
response of these diazotrophic \textit{Clostridiaceae} to temperature would be
useful towards predicting how climate change will affect the BSC nitrogen
budget. 

Although too undersampled in the environmental data sets to reach statistical
conclusions, non-heterocystous diazotrophs were found more often in below crust
samples (as opposed to BSC samples) in the \citet{Steven_2013} data and in
"light" BSC samples in the \citet{Garcia_Pichel_2013} data
(Figure~\ref{fig:rspndr_dist}). This result generates some hypotheses that are
counter to prior discussions regarding BSC diazotroph temporal dynamics (keeping
in mind this phenomenon has not been evaluated statistically). Specifically,
the transition of BSC from a light colored, developing crust to a dark, mature
crust may not mark the \textit{emergence} of diazotrophs in BSC but rather the 
\textit{transition} of the diazotroph community from heterotroph dominance to
cyanobacterial.  Additionally, the soil beneath BSC may contribute
significantly to the N budget in arid ecosystems.

It is unclear why BSC \textit{nifH} gene surveys have overwhelmingly recovered
heterocystous, cyanobacterial \textit{nifH} genes which would be in contrast to
our results. Even poorly developed BSC samples have yielded predominantly
cyanobacterial \textit{nifH} genes \citep{14766579}. And, "sub-biocrust"
samples have yielded \textit{entirely} heterocystous cyanobacterial
\textit{nifH} genes \citep{Yeager_2012}. One explanation might be that the
samples from this study are simply different in diazotrophic community
structure than those surveyed in \citet{Yeager}, \citet{14766579} and
\citet{Yeager_2012}.  Indeed, it appears that the "light" crusts used here had
a paucity of heterocystous cyanobacteria from the beginning (see above). It
should be noted that "light" and in particular "sub-biocrust" samples possess
much less heterocystous cyanobacteria in general (Figure~\ref{fig:het_steven})
so the samples used in this study are not necessarily unrepresentative of
typical poorly developed BSC simply because they are lacking heterocystous
cyanobacteria. Additionally, cyanobacterial \textit{nifH} genes would be found
in every heterocystous cyanobacterial cell, not just the heterocysts.
Therefore, the relative abundance of heterocystous cyanobacteria \textit{nifH}
in \textit{nifH} gene libraries could easily overwhelm the numbers of
\textit{nifH} genes from non-heterocystous diazotrophs. Polyploidy could
further exacerbate this bias as many cyanobacteria are estimated to have
multiple genome copies per cell \citep{Griese_2011}. In any case, the DNA-SIP
discovered diazotrophs for the "light", poorly developed BSC used in the study
were not cyanobacterial. It is unknown, however, if non-cyanobacterial diazotrophs
would be identified by $^{15}$N$_{2}$ DNA-SIP using mature BSC samples.
Regardless, our results suggest that BSC N-fixation may include a significant
non-cyanobacterial component that requires further assessment across a more
comprehensive sampling of BSC types.

\subsection{Sequencing depth}
While it is somewhat alarming how few of the putative diazotrophs found in this
study were also found by \citet{Garcia_Pichel_2013} and \citet{Steven_2013}, it
is important to point out that even next-generation sequencing efforts of BSC
16S rRNA genes have only shallowly sampled the full diversity of BSC microbes.
Rarefaction curves of all samples from \citet{Steven_2013} and
\citet{Garcia_Pichel_2013} are still sharply increasing especially for "below
crust" samples (Figure~\ref{fig:rarefaction}). Parametric richness estimates of
BSC diversity indicate the \citet{Steven_2013} and \citet{Garcia_Pichel_2013}
sequencing efforts recovered on average 40.5\% (sd. 9.99\%) and 45.5\% (sd.
11.6\%) of existing 16S OTUs from samples (inset Figure~\ref{fig:rarefaction}),
respectively. Further, the \citet{Steven_2013} and \citet{Garcia_Pichel_2013}
only share 57.6\% of total OTUs found in at least one of the studies. In fact,
this study shares more OTUs with \citet{Steven_2013}, 62.4\% of total OTUs
between both studies, than the \citet{Steven_2013} study shares with
\citet{Garcia_Pichel_2013}. 

\subsection{Temperature influences on heterocystozus cyanobacteria relative abundance} 
Although few putative diazotrophs identified by DNA-SIP were found
in the \citet{Garcia_Pichel_2013} and \citet{Steven_2013} data, we did observe
statistically significant relationships between several heterocystous
cyanobacterial OTUs with site mean annual temperature. Specifically, we found
\textit{Nostoc commune} MCT-1 and MFG-1 relative abundances were negatively
correlated with sample mean annual temperature. Additionally, it appears that
the relative abundances of \textit{Scytonema hyalinum} FGP-7A and DC-A are
positively correlated with mean annual temperature.

\citet{Yeager_2012} found \textit{nifH} gene abundance peaks in early summer
and falls in autumn. Although \citet{Yeager_2012} also experimentally
increased the ambient temperature of several BSC samples
over a long period (up to two years), changes in ambient temperature did not
influence \textit{nifH} gene abundance as measured by qPCR. We are not able to
confirm these results using the data from \citet{Garcia_Pichel_2013} which is
compositional in nature as opposed to absolute but it does appear that
temperature affects the structure of heterocystous cyanobacterial diazotroph
communities if not the absolute abundance of \textit{nifH} genes. 

\subsection{Analysis of next-generation-sequencing DNA-SIP data}
Although DNA-SIP is a powerful technique, analysis of DNA-SIP data is not
without ambiguities. One limitation is the artificial boundary in the form of a
selected adjusted p-value threshold (or false discovery rate) that marks which
OTUs we consider to be enriched in the heavy fractions of labeled CsCl
gradients (and thus have likely incorporated $^{15}$N into their DNA during the
incubation). In reality the metric we use to quantify the magnitude of an OTU's
response to a stable isotope is continuous and there is only an artificial
boundary between which OTUs appear to have "responded" and which OTUs have
unknown response. For this reason, we have presented all the OTUs that satisfy
our "response" criteria but focused on the most strongly responding OTUs. As
with any hypothesis-based statistical test, care should be taken when
interpreting the significance of results where p-values are near the selected
threshold for rejecting the null hypothesis.

\subsection{Conclusion}
It would seem unlikely given their ubiquity and abundance that heterocystous
cyanobacteria are not key contributors to the BSC N-budget. But, the putative
diazotrophs elucidated in this study and in \citet{Steppe_1996} in addition to
the N-fixation rate data presented by \citep{15643930} suggest there may be
additional and significant non-cyanobacterial BSC diazotrophs specifically
within the \textit{Clostrideaceae} and \textit{Proteobacteria}. It seems clear
that heterocystous cyanobacteria increase in abundance with BSC age
\citep{14766579}. It is less clear if this transition marks the emergence of
diazotrophy versus a re-structuring of the BSC diazotroph community from one
dominated by \textit{Firmicutes} and \textit{Proteobacteria} to one
predominantly heterocystous cyanobacteria. DNA-SIP is a valuable tool in the
molecular microbial ecologist's toolbox for identifying members of microbial
community functional guilds CITE. PCR-based surveys of diagnostic marker genes
and DNA-SIP are both used to connect microbial phylogenetic types to microbial
activities but they occupy a non-overlapping set of strengths and weaknesses.
Combined these tools can powerfully untangle connections between ecosystem
membership/structure and function. Here we supplement surveys of BSC
\textit{nifH} diversity, a diagnostic marker PCR-driven approach, with 
$^{15}$N$_{2}$ DNA-SIP, and, while we do not confirm previous results, we expand
knowledge BSC diazotroph diversity. Evaluating BSC N-fixation due to climate
change and physical disturbance requires a careful accounting of diazotrophs
including non-cyanobacterial types. 
