\section{Discussion}

\subsection{BSC Non-Cyanobacterial Diazotrophs}
BSC N-fixation has long been attributed to heterocystous cyanobacteria and molecular microbial ecology surveys of BSC \textit{nifH} gene content have been consistent with this hypothesis finding cyanobacterial \textit{nifH} types to be numerically dominant in \textit{nifH} gene libraries \cite{Yeager, 14766579, Yeager_2012}. It's possible, however, that PCR driven molecular surveys of \textit{nifH} gene content have been biased against non-heterocystous cyanobacteria (CITE GABY). Unfortunately, it is impossible to assess or quantify this bias (in either direction) without knowing the \textit{nifH} gene content \textit{de novo}. Perhaps non-PCR based molecular data such as metagenomic DNA sequence libraries will provide additional evidence with respect to the relative abundances of BSC \textit{nifH} gene types. Additionally, heterocysts (the specialized N-fixing cells along the trichome of filamentous heterocystous cyanobacteria such as \textit{Nostoc} and \textit{Scytonema}) may be overepresented with respect to non-heterocyst N-fixing cells in \textit{nifH} libraries because the heterocysts make up a fraction of the total cells along a trichome in and even the non-heterocyst cells in a trichome will possess the \textit{nifH} gene. It should also be noted that \textit{nifH} gene content is not directly extrapolable to the relative abundances of nitrogenase proteins.

We did not observe evidence for N-fixation by heterocystous cyanobacteria in the "light" crust samples used in this study. One possible explanation for our results is that the the "light", still developing BSC samples used in this study possessed less heterocystous cyanobacteria than dark mature BSC as has been observed in previous comparisons of light and dark BSC \cite{14766579}. Indeed, only 0.29\% of seqeunces from this study's DNA-SIP 16S rRNA gene seqeunce libraries were from heterocystous cyanobacteria (see results) as opposed to 15\% and 23\% of total sequences in the \citet{Steven_2013} and \citet{Garcia_Pichel_2013} data, respectively. It's difficult to compare relative abundance values from CsCl gradient fractions against environmental libraries, but, a three order of magnitude difference between the environmental librares and the CsCl gradient fractions is stark. Nonetheless, we would still expect even low abundance diazotrophs to still show evidence for N-incorporation provided sequence counts were not too sparse in heavy fractions but the OTUs defined by selected heterocystous cyanobacteria sequences presented in \citet{Yeager} all fall below the sparsity threshold (Figure X). Given the sparsity of heterocystous cyanobacteria sequences in the DNA-SIP data set it is not possible to assess whether heterocystous cyanobacteria incorporated N during the incubation.

The OTUs that did appear to incoporate N during the incubation were predominantly \textit{Proteobacteria} and \textit{Firmicutes}. The \textit{Proteobacteria} OTUs for which N-incorporation signal was strongest all shared high sequence identity (\textgreater=98.48\% sequence identity) with 16S sequences from cultivars in genera with known diazotrophs. The \textit{Firmicutes} that displayed signal for N-incorporation (predominantly \textit{Clostridiaceae}) were not closely related to any cultivars (Table X, Figure X). There appears to be a gap in culture collections for these BSC diazotrophs. As culture-based ecophysiological studies have proven useful towards explaining ecological phenomena in BSC 16S rRNA gene sequence libraries \cite{Garcia_Pichel_2013}, it would seem that these putative \textit{Clostridiaceae} diazotrophs would be prime candidates for targeted culturing efforts. Assessing the physiological response of these \textit{Clostridiaceae} to temperature would be useful towards predicting how climate change will affect the BSC nitrogen budget.   

\cite{Yeager_2012} trends in het cyano w/ temp

\subsection{Sequencing Depth}
While it is somewhat alarming how few of the putative diazotrophs found in this study were also found by \citet{Garcia_Pichel_2013} and \citet{Steven_2013}, it is important to point out that even next-generation sequencing efforts of BSC 16S rRNA genes have only shallowly sampled the full diversity of BSC microbes. Rarefaction curves of all samples from \citet{Steven_2013} and \citet{Garcia_Pichel_2013} are still sharply increasing especially for "below crust" samples (Figure X). Parametric richness estimates of BSC diversity indicate the \citet{Steven_2013} and \citet{Garcia_Pichel_2013} sequencing efforts recovered on average 40.5\% (sd. 9.99\%) and 45.5\% (sd. 11.6\%) of existing 16S OTUs from samples (Figure X), respectively. Further, the \citet{Steven_2013} and \citet{Garcia_Pichel_2013} only share 57.6\% of total OTUs found in at least one of the studies. In fact, this study shares more OTUs with \citet{Steven_2013}, 62.4\% of total OTUs between both studies, than the \citet{Steven_2013} study shares with \citet{Garcia_Pichel_2013}. 

\subsection{Temperature influences on heterocystous cyanobacteria relative abundance}
Although few putative diazotrophs identitied by DNA-SIP were found in the \citet{Garcia_Pichel_2013} and \citet{Steven_2013} data, we did make some new observations regarding the relationship of several heterocystous cyanobacterial OTUs' relationship with site mean annual temperature. Specifically, we found \textit{Nostoc commune} MCT-1 and MFG-1 relative abundances were negatively correlated with sample mean annual temperature. Additionally, it appears that \textit{Scytonema hyalinum} FGP-7A relative abundances are positively correlated with mean annual temperature. Although these OTUs do not represent the full diversity of heterocystous cyanobacteria in the combined data, predicting climate change effects on BSC N-fixation.

\subsection{Analysis of next-generation-sequencing DNA=SIP data}
Although DNA-SIP is a powerful technique, analysis of DNA-SIP data is not without ambiguities. One limitation is the artificial boundary in the form of a selected adjusted p-value threshold (or false discovery rate) that marks which OTUs we consider to have incorporated N into their DNA during the incubation. In reality the metric we use to quantify the magnitude of an OTU's response to a stable isotope is continuous and there is not a non-artificial boundary for this metric between what OTUs appear to have "responded" to which OTUs have unknown response. For this reason, we have presented the OTUs that satisfy our "response" criteria while focusing on the most strongly responding OTUs. As with any hypthesis-based statistical test care should be taken when interpreting the significance of results where p-values are near the selected "significance" threshold for rejecting the null hypothesis.

Heterocyst abundance along trichome
DNA-SIP continuous result. "Boundary" in results that denotes responders is artificial. Eco-phyiological studies with cultivars to confirm results CITE THE BELNAP STUDY.

