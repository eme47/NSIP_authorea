\section{Introduction}
Biological soil crusts (BSC) are specialized microbial mat communitites that
form at the soil surface in arid environments and fill a variety of important
ecological functions. BSC occupy plant interspaces and cover
a wide, global geographic range \citep{garcia2003estimates}. The ground cover
of BSC on the Colorado Plateau has been measured as high as 80\% by remote
sensing \citep{karnieli2001}. The global biomass of BSC \textit{Cyanobacteria}
alone is estimated at 54 x 10$^{12}$ g C \citep{garcia2003estimates}. BSC 
are responsible for significant nitrogen (N) flux (for review of BSC
N$_{2}$-fixation see \citet{belnap2003}).  N$_{2}$-fixation represents the
dominant source of new ecosystem N in more than 80\% of BSC from diverse sites
across North America, Africa, and Australia, while atmospheric N deposition was
a dominant source of N in only a minority of sites \citep{Evans_1999}. The
presence of BSC is positively correlated with vascular plant survival due in
part to BSC ecosystem N contributions (for review of BSC-vascular plant
interactions see \citet{BelnapVascular}). Climate change and disturbance could
alter BSC microbial community structure/membership and possibly BSC diazotroph
diversity and N$_{2}$-fixation.

BSC N$_{2}$-fixation rate studies (typically employing the acetylene reduction
assay (ARA)) have explored BSC diazotroph activity across various ecological
gradients. Reported BSC N$_{2}$-fixation rates vary significantly across
samples and studies \citep{Evans_2001}.  The reasons for inter-site and
inter-study variability are complex and likely include the spatial
heterogeneity of BSC \citep{Evans_2001} and the impact of recent environmental
change on N$_{2}$-fixation rates (see \citet{Belnap_2001} for discussion).
Moreover, the ARA assay is subject to methodological artifacts that can
complicate making robust comparisons across sample types that differ in
physical and biological characteristics (see \citet{Belnap_2001} for review).
Nonetheless, N$_{2}$-fixation rates are consistently higher in mature BSC than
in early successional BSC \citep{Belnap_2002, 14766579}. This difference
may be due to the proliferation of heterocystous \textit{Cyanobacteria} in
mature BSC and is consistent with the theory that heterocystous
\textit{Cyanobacteria} provide the main source of fixed-N.
Alternatively, the N$_{2}$-fixation rate differences between early
successional and mature BSC might be attributable to methodological artifacts.
For instance, N$_{2}$-fixation in mature BSC is maximal at the crust surface
(coincident with heterocystous cyanobacteria) while it is maximal below the
crust surface in early successional BSC \citep{15643930}. Diffusional
limitation can cause ARA to underestimate N$_{2}$-fixation which
occurs below the crust surface and as a result ARA can systematically
underestimate rates of N$_{2}$-fixation in early successional BSC
\citep{15643930}. Diffusion would not be an issue when measuring
N$_{2}$-fixation rates in mature crusts as nitrogenase activity peaks near the
surface.  The difference between N$_{2}$ fixation rates of early
successional and mature BSC were not statistically significant when
total rates were estimated by integrating multiple rates from
thin (1-3mm) slices along BSC depth profiles (as opposed to 
intact cores) \citep{15643930}.

Molecular studies of BSC microbial diversity include explorations of the BSC
microbial community vertical profile \citep{Garcia_Pichel_2003}, BSC
\textit{nifH} gene content surveys (e.g. \citet{14766579}, \citet{Yeager_2012},
\citet{Yeager} and \citet{Steppe_1996}), and next-generation-sequencing (NGS)
enabled studies of BSC SSU rRNA genes across wide geographic ranges
\citep{Garcia_Pichel_2013, Steven_2013}. Early successional  BSC are often
described as "light" in appearance relative to "dark" mature BSC
\citep{Belnap_2002, 14766579}. Mature BSC possess greater numbers of
heterocystous \textit{Cyanobacteria} (i.e \textit{Scytonema},
\textit{Spirirestis}, and \textit{Nostoc} \citep{Yeager, Yeager_2012}) than
developing BSC but both early successional and mature BSC are dominated by
non-heterocystous \textit{Cyanobacteria} (\textit{Microcoleus vaginatus} or
\textit{M. steenstrupii}) \citep{14766579, Garcia_Pichel_2013}.  Heterocystous
\textit{Cyanobacteria} are the numerically dominant BSC diazotrophs in
\textit{nifH} clone libraries \citep{Yeager, 14766579, Yeager_2012}.
Eighty-nine perent of 693 \textit{nifH} sequences derived from Colorado Plateau and
New Mexico BSC samples were described as heterocystous cyanobacterial
(non-cyanobacterial \textit{nifH} sequences were largely attributed to alpha-
and beta- \textit{Proteobacteria}) \citep{Yeager}. However, an early survey of
Colorado Plateau BSC \textit{nifH} diversity recovered \textit{nifH} genes
related to \textit{Gammaproteobacteria} as well as a clade that included
\textit{nifH} genes from the anaerobes \textit{Clostridium pssteurianum},
\textit{Desulfovibrio gigas} and \textit{Chromatium buderi} \citep{Steppe_1996},

The influence of microbial community membership and structure on BSC
N$_{2}$-fixation is an ongoing research question \citep{Belnap28062013}. While
the presence/abundance of heterocystous \textit{Cyanobacteria} has been
proposed as the mechanism behind increased N$_{2}$-fixation in mature BSC, it
is unclear if mature BSC actually fix more N than early successional BSC (see
\citet{15643930}). More studies are necessary to elucidate the microbial
membership influence on BSC N$_{2}$-fixation and to determine if heterocystous
\textit{Cyanobacteria} are the only keystone diazotrophs. The first step in
defining structure function relationships with respect to N$_{2}$-fixation is a
full accounting of BSC diazotrophs. Towards this end we conducted
$^{15}$N$_{2}$ DNA stable isotope probing (DNA-SIP) experiments with early
successional Colorado Plateau BSC.  DNA-SIP with $^{15}$N$_{2}$ has not been
previously attempted with BSC. DNA-SIP provides an accounting of \textit{active}
diazotrophs whereas \textit{nifH} clone libraries account for microbes with the
genomic potential for N$_{2}$-fixation.  Further, we investiage the
distribution of these active diazotrophs in BSC NGS microbial diversity surveys
over a range of spatial scales and soil types \citep{Garcia_Pichel_2013,
Steven_2013}. 
