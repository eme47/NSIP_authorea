\section{Introduction}

%What are soil crusts?
%Ecological importance/function

Biological soil crusts (BSC) are a dessicated cyanobacterial mat-like surface layer atop arid soil. Millmeters in depth, BSC are found in arid ecosystem plant interspaces and cover a wide geographic range thoughout the world \cite{garcia2003estimates}. 
%Cover on Colorado Plateau. 
BSC are responsible for significant arid ecosystem N flux (CITE). MORE BACKGROUND FROM GARCIA-PICHEL SCIENCE PAPER AND GARCIA-PICHEL BIOMASS ESTIMATE PAPER. Mature BSC possess greater numbers of heterocystous cyanobacteria (e.g. \textit{Nostoc, Syctonema}) than developing BSC but both young and old BSC are dominated by the non-heterocystous cyanobacterium \textit{Microcoleus vaginatus}. Young BSC are often described as "light" in appearance relative to "dark" mature BSC, CITE.

%Why characterize diazotrophs?
%Conclusions from previous studies
N$_{2}$ fixation in BSC has been widely studied via culture-based and culture independent molecular techniques (e.g. characterization \textit{nifH} gene diversity, CITE) and analytical methods (e.g. acetylene reduction assays to determine N$_{2}$ fixation CITE). Molecular studies have elucidated a vertically structured community CITE within the top X mm of BSC and have characterized the diversity of potential diazotrophs from \textit{nifH} gene sequence libraries CITE. Although an early study of \textit{nifH} diversity recovered genes from a Colorado Plateau BSC that appeared to be derived from the \textit{Gammaproteobacteria} as well as a clade that included \textit{nifH} genes from the anaerobes \textit{Clostridium pssteurianum}, \textit{Desulfovibrio gigas} and \textit{Chromatium buderi}, subsequent studies have found heterocystous cyanobacteria to be the numerically dominant BSC diazotrophs CITE. 

Measured N$_{2}$ BSC fixation rates vary widely likely reflecting the spatial heterogeneity of BSC but older, dark BSC N$_{2}$ rates have consistently been measured higher than for younger, light BSC CITE. This difference may be due to the proliferation of heterocystous cyanobacteria in older mats. Alternatively, the N fixation rate differences between young and old BSC might be attributable to methodological artifacts. For instance, \citet{15643930} show that N$_{2}$ fixation rates measured from intact cores of developing BSC may be artifcatually low due to delayed acetylene/ethylene diffusion through the crust in a typical acetylene reduction assay incubation timeframe. When total N$_{2}$ fixation rates were calculated by integrating N fixation rates of 1-3 mm depth BSC core slices (thus mitigating flux limitations), N$_{2}$ fixation rate differences between young and old BSCs were not statistically significant \cite{15643930}.

%Heterocystous types
%Depth profile of nitrogen fixation
%suggests non-phototroph N-fixation
%Conclusion of this study
%method considerations 
To further probe the diversity of diazotrophs in BSC we conducted $^{15}$N DNA stable isotope probing (SIP) experiments with developing Colorado Plateau biological soil crusts (BSC). Although molecular characterizations of BSC \textit{nifH} diversity in other studies have yielded predominantly heterocystous cyanobacteria \textit{nifH} genes CITE, microbes from young, developing BSC that incorporated $^{15}$N into DNA as determined by DNA-SIP in this study were not cyanobacteria but included heterotrophic diazotrophs in the \textit{Gammaproteobacteria}, \textit{Clostridiaceae} and \textit{Deltaproteobacteria}.


