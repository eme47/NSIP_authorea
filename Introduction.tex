\section{Introduction}

%What are soil crusts?
%Ecological importance/function

Biological soil crusts (BSC) are dessicated cyanobacterial mat-like topsoil crusts. Millmeters in depth, BSC are found in arid soil plant interspaces and cover a wide geographic range CITE. BSC are responsible for significant N flux in arid ecosystems (CITE). Early BSC development is characterized by a period of non-heterocystous cyanobacteria predominance (specific). Older BSC, while still domminated by the non-heterocystous cyanobacterium \dots, include heterocystsous types (e.g. \textit{Nostoc, Syctonema}) as well. Young BSC are often described as "light" inappearance relative to "darker" mature BSC, CITE.

%Why characterize diazotrophs?
%Conclusions from previous studies
N$_{2}$ fixation in BSC has been widely studied via culture-based and culture independent molecular techniques (e.g. characterization of SSU rRNA gene and \textit{nifH} gene content, CITE) and analytical methods (e.g. acetylene reduction assays to determine N$_{2}$ fixation rates and microsensor profiles of pO$_{2}$ with depth, CITE). Molecular studies have elucidated a vertically structured community CITE within the top X mm of BSC and have characterized the diversity of potential diazotrophs from nifH gene sequence libraries CITE. Although an early study of nifH diversity recovered genes from a Colorado Plateau BSC that appeared to be derived from Clostrideae (right?) and Deltaproteobacteria CITE, subsequent studies have found heterocystous cyanobacteria to be the numerically dominant BSC diazotrophs CITE. 

Measured N$_{2}$ BSC fixation rates vary widely likely reflecting the spatial heterogeneity of BSC but older, dark BSC rates have consistently been measured higher than younger, light BSC CITE. This difference may be due to the appearance of heterocystous cyanobacteria in older mats. Alternatively, the N fixation rate differences between young and old BSC might be attributable to methodological artifacts. For instance, \citet{15643930} show that N$_{2}$ fixation rates measured from intact cores of developing BSC may be artifcatually low due to delayed acetylene/ethylene diffusion in a typical acetylene reduction assay incubation timeframe. When total N fixation rates were integrated over N fixation rates of BSC core subsections along the depth of the cores, N fixation rate differences between young and old BSCs were not statistically significant \cite{15643930}.

%Heterocystous types
%Depth profile of nitrogen fixation
%suggests non-phototroph N-fixation
%Conclusion of this study
%method considerations 
To further probe the diversity of diazotrophs in BSC we conducted $^{15}$N DNA stable isotope probing (SIP) experiments with developing Colorado plateau biological soil crusts (BSC). Although molecular characterizationss of BSC \textit{nifH} diversity in other studies have yielded predominantly heterocystous cyanobacteria-like \textit{nifH} genes CITE, microbes from young, developing BSC that incorporated $^{15}$N into DNA as determined by DNA-SIP were not cyanobacteria but rather heterotrophic diazotrphs in (list phyla? Taxonomy?).


