\section{Introduction}

%What are soil crusts?
%Ecological importance/function

Biological soil crusts (BSC) are a microbial mat-like surface layer in arid
soil. Millmeters in depth, BSC are found in plant interspaces and cover a wide,
global geographic range \citep{garcia2003estimates}. The ground cover of BSC on
the Colorado Plateau has been measured as high as 80\% by remote sensing
\citep{karnieli2001}. The global biomass of BSC cyanobacteria alone is estimated
at 54 x 10$^{12}$ g C \citep{garcia2003estimates}. BSC play important roles in
arid ecosystem productivity and are responsible for significant nitrogen (N)
flux (for review of BSC N-fixation see \citet{belnap2003}). For example,
\citet{Evans_1999} found approximately five times as many BSC samples from
sites in North America, Africa and Australia had $\delta^{15}$N values
indicative of high N-fixation input relative to the number of samples where
$\delta^{15}$N indicated N input was predominantly from atmospheric deposition.
The presence of BSC is positively correlated with vascular plant
survival due in part to BSC ecosystem N contributions (for review of
BSC-vacular plant interactions see \citet{BelnapVascular}).

Molecular studies of BSC microbial diversity include explorations of the BSC
microbial community vertical profile \citep{Garcia_Pichel_2003}, BSC
\textit{nifH} gene content surveys (e.g. \citet{14766579}, \citet{Yeager_2012},
\citet{Yeager} and \citet{Steppe_1996}), and next-generation-sequencing (NGS)
enabled studies of BSC SSU rRNA gene content across wide geographic ranges
\citep{Garcia_Pichel_2013, Steven_2013}. \citet{Garcia_Pichel_2003} found that
BSC microbial diversity is organized vertically, likely as the result of
vertically oriented environmental gradients (e.g. light and oxygen).
\textit{nifH} surveys have been conducted across BSC development stages
\citep{14766579}, as well as across seasons, temperatures and precipitation
gradients \citep{Yeager_2012}. Mature, more fully developed BSC possess greater
numbers of heterocystous cyanobacteria (e.g. \textit{Nostoc, Syctonema}) than
developing BSC but both young and old BSC are dominated by non-heterocystous
cyanobacteria (\textit{Microcoleus vaginatus} or \textit{M. steenstrupii})
\citep{14766579, Garcia_Pichel_2013}. Young or recently disturbed BSC are often
described as "light" in appearance relative to "dark" mature BSC
\citep{Belnap_2002, 14766579}.  Although an early study of Colorado Plateau BSC
\textit{nifH} diversity presented \textit{nifH} genes related to
\textit{Gammaproteobacteria} as well as a clade that included \textit{nifH}
genes from the anaerobes \textit{Clostridium pssteurianum},
\textit{Desulfovibrio gigas} and \textit{Chromatium buderi}, subsequent studies
have found heterocystous cyanobacteria to be the numerically dominant BSC
diazotrophs \citep{Yeager, 14766579, Yeager_2012}. Specifically,
\citet{Yeager}--in a study of overall BSC \textit{nifH} diversity--categorized
89\% of 693 \textit{nifH} sequences derived from Colorado Plateau and New
Mexico BSC samples as heterocystous cyanobacterial (non-cyanobacterial
\textit{nifH} sequences were largely attributed to alpha- and beta-
\textit{proteobacteria}). The heterocystous cyanobacterial BSC diazotrophs fall
into three genera, \textit{Scytonema}, \textit{Spirirestis}, and
\textit{Nostoc} \citep{Yeager, Yeager_2012}. Studies of BSC microbial diversity
over broad geographic ranges have elucidated how soil parent material
correlates to above and below crust microbial community membership and
structure \citep{Steven_2013} and that the predominant BSC cyanobacterium shifts
from \textit{M. vaginatus} to \textit{M. steenstrupii} with increasing mean
annual temperature \citep{Garcia_Pichel_2013}.

%Why characterize diazotrophs?
%Conclusions from previous studies
BSC N-fixation rate studies (typically employing the acetylene reduction assay
(ARA)) have explored BSC diazotroph activity across various ecological
gradients. Reported BSC N-fixation rates vary significantly \citep{Evans_2001}.
The reasons for this variability are complex and likely include the spatial
heterogeneity of BSC \citep{Evans_2001} and the impact of recent environmental
conditions on N-fixation rates (see \citet{Belnap_2001} for discussion).
Moreover, the ARA assay is subject to methodological artifacts that preclude
cross-study and possibly intra-study but inter-environment type comparisons
(see \citet{Belnap_2001} for review). Despite the general BSC N-fixation rate
measurement variability, mature, dark BSC N-fixation rates have been measured
higher than N-fixation rates for younger, light BSC \citep{Belnap_2002,
14766579}. This difference may be due to the proliferation of heterocystous
cyanobacteria in older mats and is consistent with the theory that
heterocystous cyanobacteria are the primary BSC diazotrophs. Alternatively, the
N-fixation rate differences between young and old BSC might be attributable to
methodological artifacts. For instance, \citet{15643930} show that N-fixation
rates measured from intact cores of developing BSC may be artifcatually low due
to delayed acetylene/ethylene diffusion through the crust in a typical ARA
incubation timeframe. When total N-fixation rates were calculated by
integrating N-fixation rates over 1-3 mm depth slices along the full
BSC core (thus mitigating ethene/acetylene flux limitations), N-fixation rate
differences between developing and mature BSC were not statistically significant
\citep{15643930}.

%Heterocystous types
%Depth profile of nitrogen fixation
%suggests non-phototroph N-fixation
%Conclusion of this study
%method considerations 
The influence of microbial community membership and structure on BSC N-fixation
is an ongoing research question \citep{Belnap28062013}. While the
presence/abundance of heterocystous cyanobacteria has been proposed as the
underlying microbial membership influence on increased N-fixation in mature
BSC, it is unclear if the premise that mature BSC fix more N is always correct
(see \citet{15643930}). More studies are necessary to elucidate the microbial
membership influence on BSC N-fixation and to determine if heterosystous
cyanobaceria are the only keystone diazotrophs. To further probe the diversity
of diazotrophs in BSC we conducted $^{15}$N DNA stable isotope probing
(DNA-SIP) experiments with light, developing Colorado Plateau BSC. Although
molecular characterizations of BSC \textit{nifH} diversity in other studies
have yielded predominantly heterocystous cyanobacterial \textit{nifH} genes, in
ths study microbes from young, developing BSC that incorporated N from
N$_{2}$ into DNA as determined by DNA-SIP were not cyanobacteria but
members of the \textit{Gammaproteobacteria}, \textit{Clostridiaceae} and
\textit{Deltaproteobacteria}. Further, we track the distribution of putative
diazotrophs uncovered in this study in addition to heterocystous
cyanobacteriadia studied by \citet{14766579}, \citet{Yeager} and
\citet{Yeager_2012} through collections of NGS SSU rRNA libraries from BSC
microbial diversity surveys over a range of spatial scales and soil types
\citep{Garcia_Pichel_2013, Steven_2013}.
