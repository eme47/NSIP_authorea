\section{Introduction}
Biological soil crusts (BSC) are specialized microbial mat communitites that form at the soil surface in arid environmets and fill a variety of important ecological functions in arid ecosystems. BSC occupy plant interspaces and cover a
wide, global geographic range \citep{garcia2003estimates}. The ground cover of
BSC on the Colorado Plateau has been measured as high as 80\% by remote sensing
\citep{karnieli2001}. The global biomass of BSC \textit{Cyanobacteria} alone is
estimated at 54 x 10$^{12}$ g C \citep{garcia2003estimates}. BSC play important
roles in arid ecosystem productivity and are responsible for significant
nitrogen (N) flux (for review of BSC N-fixation see \citet{belnap2003}). For
example, N-input via N-fixation versus atmospheric depositon was predominant in
five times as many BSC samples from North America, Africa and Australia
\citep{Evans_1999}.  The presence of BSC is positively correlated with vascular
plant survival due in part to BSC ecosystem N contributions (for review of
BSC-vascular plant interactions see \citet{BelnapVascular}). Climate change and
disturbance alter BSC microbial community structure and membership and therefore
can alter diazotroph diversity and the BSC N-budget.

BSC N-fixation rate studies (typically employing the acetylene reduction assay
(ARA)) have explored BSC diazotroph activity across various ecological
gradients. Reported BSC N-fixation rates vary significantly \citep{Evans_2001}.
The reasons for this variability are complex and likely include the spatial
heterogeneity of BSC \citep{Evans_2001} and the impact of recent environmental
conditions on N-fixation rates (see \citet{Belnap_2001} for discussion).
Moreover, the ARA assay is subject to methodological artifacts that preclude
cross-study and possibly intra-study but inter-environment type comparisons
(see \citet{Belnap_2001} for review). Nonetheless, mature BSC N-fixation rate
measurements have been higher than younger, developing BSC N-fixation rate
measurements \citep{Belnap_2002, 14766579}. This difference may be due to the
proliferation of heterocystous \textit{Cyanobacteria} in older mats and is
consistent with the theory that heterocystous \textit{Cyanobacteria} are the
primary BSC diazotrophs.  Alternatively, the N-fixation rate differences
between young and old BSC might be attributable to methodological artifacts.
For instance, \citet{15643930} show that N-fixation rates peak at a lower depth
in developing BSC as compared to mature BSC. When N-fixation is measured from
intact cores of developing BSC the measurement may be artifactually low due to
delayed acetylene/ethylene diffusion through the crust to and from the peak
N-fixation rate depth in a typical ARA incubation timeframe. Diffusion would
not be an issue when measuring N-fixation rates in mature crust as nitrogenase
activity peaks near the surface. When total N-fixation rates were calculated by
integrating rates over 1-3 mm depth slices along full BSC cores (thus
mitigating ethene/acetylene flux limitations), N-fixation rate differences
between developing and mature BSC were not statistically significant
\citep{15643930}.

Molecular studies of BSC microbial diversity include explorations of the BSC
microbial community vertical profile \citep{Garcia_Pichel_2003}, BSC
\textit{nifH} gene content surveys (e.g. \citet{14766579}, \citet{Yeager_2012},
\citet{Yeager} and \citet{Steppe_1996}), and next-generation-sequencing (NGS)
enabled studies of BSC SSU rRNA gene content across wide geographic ranges
\citep{Garcia_Pichel_2013, Steven_2013}. 
%\citet{Garcia_Pichel_2003} found that
%BSC microbial diversity is organized vertically, likely as the result of
%vertically oriented environmental gradients (e.g. light and oxygen).
\textit{nifH} surveys have been conducted across BSC development stages
\citep{14766579}, as well as across seasons, temperatures and precipitation
gradients \citep{Yeager_2012}. Mature, more fully developed BSC possess greater
numbers of heterocystous \textit{Cyanobacteria} (e.g. \textit{Nostoc,
Syctonema}) than developing BSC but both young and old BSC are dominated by
non-heterocystous \textit{Cyanobacteria} (\textit{Microcoleus vaginatus} or
\textit{M. steenstrupii}) \citep{14766579, Garcia_Pichel_2013}. Young or
recently disturbed BSC are often described as "light" in appearance relative to
"dark" mature BSC \citep{Belnap_2002, 14766579}. Heterocystous
\textit{Cyanobacteria} are the numerically dominant BSC diazotrophs
\citep{Yeager, 14766579, Yeager_2012} in \texit{nifH} clone libraries although
an early survey of Colorado Plateau BSC \textit{nifH} diversity recovered
\textit{nifH} genes related to \textit{Gammaproteobacteria} as well as a clade
that included \textit{nifH} genes from the anaerobes \textit{Clostridium
pssteurianum}, \textit{Desulfovibrio gigas} and \textit{Chromatium buderi},
Specifically, \citet{Yeager}--in a study of overall BSC \textit{nifH}
diversity--categorized 89\% of 693 \textit{nifH} sequences derived from
Colorado Plateau and New Mexico BSC samples as heterocystous cyanobacterial
(non-cyanobacterial \textit{nifH} sequences were largely attributed to alpha-
and beta- \textit{proteobacteria}). The heterocystous cyanobacterial BSC
diazotrophs fall into three genera, \textit{Scytonema}, \textit{Spirirestis},
and \textit{Nostoc} \citep{Yeager, Yeager_2012}. 
%Studies of BSC microbial diversity
%over broad geographic ranges have elucidated how soil parent material
%correlates to above and below crust microbial community membership and
%structure \citep{Steven_2013} and that the predominant BSC cyanobacterium
%shifts from \textit{M. vaginatus} to \textit{M. steenstrupii} with increasing
%mean annual temperature \citep{Garcia_Pichel_2013}.

The influence of microbial community membership and structure on BSC N-fixation
is an ongoing research question \citep{Belnap28062013}. While the
presence/abundance of heterocystous \textit{Cyanobacteria} has been proposed as
the mechanism behind increased N-fixation in mature BSC, it is unclear if
mature BSC actually fix more N (see \citet{15643930}). More studies are
necessary to elucidate the microbial membership influence on BSC N-fixation and
to determine if heterocystous \textit{Cyanobacteria} are the only keystone
diazotrophs. The first step in defining structure function relationships with
respect to N-fixation is a full accounting of BSC diazotrophs. Towards this end
we conducted $^{15}$N$_{2}$ DNA stable isotope probing (DNA-SIP) experiments
with light, developing Colorado Plateau BSC. DNA-SIP with $^{15}$N$_{2}$ has not
been attempted with BSC. DNA-SIP would provides an accounting of 
\textit{active} diazotrophs whereas \textit{nifH} clone libraries account for 
microbes with the genomic potential for N-fixation. Further, we track the
distribution of putative diazotrophs uncovered in this study through
collections of NGS SSU rRNA libraries from BSC microbial diversity surveys over
a range of spatial scales and soil types \citep{Garcia_Pichel_2013,
Steven_2013}. 
