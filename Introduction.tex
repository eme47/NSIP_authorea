\section{Introduction}

%What are soil crusts?
%Ecological importance/function

Biological soil crusts (BSC) are an often dessicated cyanobacterial mat-like surface layer in arid soil. Millmeters in depth, BSC are found in plant interspaces and cover a wide geographic range thoughout the world \cite{garcia2003estimates}. BSC ground cover on the Colorado Plateau has been measured as high as 80\% by remote sensing \cite{karnieli2001} and global biomass of BSC cyanobacteria alone is estimated at 54 x 10$^{12}$ g C \cite{garcia2003estimates}. BSC are responsible for significant arid ecosystem N flux (for review of BSC N$_{2}$ fixation see \citet{belnap2003}). \citet{Evans_1999} found approximately five times as many soil crust samples from sites in North America, Africa and Australia had $\delta^{15}$N values indicative of high N-fixation input relative to the number of samples where N-input was likely dominated by atmospheric deposition. Something about vascular plants N-content, N-limitation etc to wrap this up...

BSC microbial diversity studies include explorations of vertical BSC diversity profiles CITE, BSC development CITE and BSC nifH gene content CITE. \textbf{Summarize Garcia-Pichel vertical profile study}. Mature BSC possess greater numbers of heterocystous cyanobacteria (e.g. \textit{Nostoc, Syctonema}) than developing BSC but both young and old BSC are dominated by the non-heterocystous cyanobacterium \textit{Microcoleus vaginatus} CITE. Young BSC are often described as "light" in appearance relative to "dark" mature BSC, CITE.  Although an early study of \textit{nifH} diversity recovered genes from a Colorado Plateau BSC that appeared to be derived from the \textit{Gammaproteobacteria} as well as a clade that included \textit{nifH} genes from the anaerobes \textit{Clostridium pssteurianum}, \textit{Desulfovibrio gigas} and \textit{Chromatium buderi}, subsequent studies have found heterocystous cyanobacteria to be the numerically dominant BSC diazotrophs CITE. \textbf{summarize nifH studies. Focus on cyano are numerically dominant results}. 

%Why characterize diazotrophs?
%Conclusions from previous studies
N$_{2}$ fixation in BSC has been widely studied via culture-based and culture independent molecular techniques (e.g. characterization \textit{nifH} gene diversity, CITE) and analytical methods (e.g. acetylene reduction assays to determine N$_{2}$ fixation CITE). Molecular studies have elucidated a vertically structured community CITE within the top X mm of BSC and have characterized the diversity of potential diazotrophs from \textit{nifH} gene sequence libraries CITE. 

Measured N$_{2}$ BSC fixation rates vary widely CITE. The source of this variability is unclear and likely includes the spatial heterogeneity of BSC, the impact on recent environmental condtions on N-fixation rates, and variability in C2H2 conversion ratios CITE. Despite the confounding overall N-fixation rate measurement variability older, dark BSC N$_{2}$ rates have consistently been measured higher than for younger, light BSC CITE. This difference may be due to the proliferation of heterocystous cyanobacteria in older mats and is consistent with the assumption that heterocystous cyanobacteria are the primary BSC diazotrophs (see CITE). Alternatively, the N fixation rate differences between young and old BSC might be attributable to methodological artifacts. For instance, \citet{15643930} show that N$_{2}$ fixation rates measured from intact cores of developing BSC may be artifcatually low due to delayed acetylene/ethylene diffusion through the crust in a typical acetylene reduction assay incubation timeframe. When total N$_{2}$ fixation rates were calculated by integrating N fixation rates of 1-3 mm depth BSC core slices (thus mitigating flux limitations), N$_{2}$ fixation rate differences between young and old BSCs were not statistically significant \cite{15643930}.

%Heterocystous types
%Depth profile of nitrogen fixation
%suggests non-phototroph N-fixation
%Conclusion of this study
%method considerations 
To further probe the diversity of diazotrophs in BSC we conducted $^{15}$N DNA stable isotope probing (SIP) experiments with developing Colorado Plateau biological soil crusts (BSC). Although molecular characterizations of BSC \textit{nifH} diversity in other studies have yielded predominantly heterocystous cyanobacteria \textit{nifH} genes CITE, microbes from young, developing BSC that incorporated $^{15}$N into DNA as determined by DNA-SIP in this study were not cyanobacteria but included heterotrophic diazotrophs in the \textit{Gammaproteobacteria}, \textit{Clostridiaceae} and \textit{Deltaproteobacteria}.


