We conducted $^{15}$N DNA stable isotope probing (SIP) experiments to investigate diazotroph diversity in developing Colorado plateau biological soil crusts (BSC). Although molecular characterizations of BSC \textit{nifH} diversity have yielded predominantly heterocystous cyanobacteria-like \textit{nifH} genes, microbes that appear to incorporate $^{15}$N into DNA as determined by DNA-SIP were not cyanobacteria.

BSC cover a wide geographic range and are a major node with respect to N flux to and from arid ecosystems. Early BSC development is characterized by a period of non-heterocystous cyanobacteria predominance. Older BSC, while maintaining a significant collection onf non-heterocystous cyanobacteria, include heterocystsous types (e.g. \textit{Nostoc, Syctonema}. Measured N$_{2}$ fixation rates are generally higher in older BSC than younger. However, \citet{15643930} show that N$_{2}$ fixation rates measured from intact cores of developing BSC may be artifcatually low due to delayed acetylene/ethylene diffusion. 


