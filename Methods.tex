Methods

Field sites

Soil crust incubation

DNA from each sample was extracted using a MoBio Kit (following manufacturer’s protocol, but substituting a 2 minute bead beating for the vortexing step), and then gel purified. Extracts were quantified using PicoGreen nucleic acid quantification dyes (Molecular Probes). 

 Gradient density centrifugation of DNA was undertaken in 6 mL polyallomer centrifuge tubes in a TLA-110 fixed angle rotor (both Beckman Coulter) in CsCl gradients with an average density of 1.725 g mL-1.  Average density for all prepared gradients was checked with an AR200 refractometer before runs. Between 2.5- 5 μg of DNA extract was added to the CsCl solution, and gradients were run under conditions of 20°C for 67 hours at 55,000 rpm (Lueders et al., 2004). Centrifuged gradients were fractionated from bottom to top in 36 equal fractions of 100 μL, using a displacement technique similar to Manefield et al. (2002). The density of each fraction was determined using a refractometer. DNA in each fraction was desalted through four washes with 300 μL TE per fraction. 

Bacterial and archaeal 16s rRNA genes from each fraction were quantified through real-time PCR, using primers Ba519f/Ba907r (Stubner, 2002).  Each 25 μL reaction contained 1X Quantifast SYBR Green Master Mix (Qiagen), 0.3 μM of each primer, and 1 μL of a 1:100 dilution of fraction DNA. Thermal cycling occurred with an initial denaturation step of 10 minutes at 95°C, followed by 40 cycles of amplification (15s at 95°C, then 60s at 60°C). After each run, a melt curve was measured and recorded between 60°C and 95°C. Quantification was achieved through use of a dilution series (108-101 copies/μL) from nearly full length 16s rRNA gene amplicons from pure culture DNA of K. pneumoniae. 

Barcoded PCR of bacterial and archaeal 16s rRNA genes, in preparation for 454 Pyrosequencing, was carried out using primer set Ba515F/Ba806R. The primer Ba806R contained an 8 bp error-correcting barcode sequence, a TC linker, and a Roche 454 B sequencing adaptor, while the primer Ba515F contained the Roche 454 A sequencing adaptor (Hamady, 2008; Fierer et al., 2008). Each 25 μL reaction contained 1x PCR Gold Buffer (Roche), 2.5 mM MgCl2, 200 μM of each of the four dNTPs (Promega), 0.5 mg/mL BSA (New England Biolabs), 0.3 μM of each primers, 1.25 U of Amplitaq Gold (Roche), and 8 μL of template. Template for each sample was added at normalized amounts in an attempt to prevent chimera formation, and each sample was amplified in triplicate. Thermal cycling occurred with an initial denaturation step of 5 minutes at 95°C, followed by 40 cycles of amplification (20s at 95°C, 20s at 53°C, 30s at 72°C), and a final extension step of 5 min at 72°C. Triplicate amplicons were pooled and purified using Agencourt AMPure PCR purification beads, following manufacturer’s protocol. Once cleaned, amplicons were quantified using PicoGreen nucleic acid quantification dyes (Molecular Probes) and pooled together in equimolar amounts. Samples were sent to the Environmental Genomics Core Facility at the University of South Carolina (now Selah Genomics) to be run on a Roche FLX 454 pyrosequencing machine. 

Data Analysis 
