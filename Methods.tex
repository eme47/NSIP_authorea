\section{Materials and Methods}
\subsection{Field site and sample description}
Samples were taken from Green Butte, Arizona as previously described (site CP3,
\citet{BERALDI_CAMPESI_2009}). All samples were from light crusts as described
by \cite{15643930}.

\subsection{Soil crust incubation}
Light crust samples (37.5 cm$^{2}$, average mass 35 g) were incubated in sealed
chambers under controlled atmosphere and in the light for 4 days. Crusts were
dry prior to time zero and were wetted at initiation of experiment. Treatments
included control air (unenriched headspace) and enriched air (\textgreater98\%
atom $^{15}$N$_{2}$) headspace. Samples were taken at 2 days and 4 days
incubation.  Acetylene reduction rates were measured daily. DNA was extracted
from 1 g of crust.

\subsection{DNA extraction}
DNA from each sample was extracted using a MoBio PowerSoil DNA Isolation Kit (following manufacturer’s protocol, but substituting a 2 minute bead beating for the vortexing step), and then gel purified. Extracts were quantified using PicoGreen nucleic acid quantification dyes (Molecular Probes). 
\subsection{DNA-SIP}
 Gradient density centrifugation of DNA was undertaken in 6 mL polyallomer
 centrifuge tubes in a TLA-110 fixed angle rotor (both Beckman Coulter) in CsCl
 gradients with an average density of 1.725 g/mL.  Average density for all
 prepared gradients was checked with an AR200 refractometer before runs.
 Between 2.5-5 $\mu$g of DNA extract was added to the CsCl solution, and
 gradients were run under conditions of 20°C for 67 hours at 55,000 rpm
 (Lueders et al., 2004). Centrifuged gradients were fractionated from bottom to
 top in 36 equal fractions of 100 $\mu$L, using a displacement technique
 similar to Manefield et al. (2002). The density of each fraction was
 determined using a refractometer. DNA in each fraction was desalted through
 four washes with 300 $\mu$L TE per fraction.  \subsection{PCR, library
 normalization and DNA sequencing} Barcoded PCR of bacterial and archaeal 16S
 rRNA genes, in preparation for 454 Pyrosequencing, was carried out using
 primer set 515F/806R \citep{21349862}.  The primer 806R contained an 8 bp
 barcode sequence, a "TC" linker, and a Roche 454 B sequencing adaptor, while
 the primer 515F contained the Roche 454 A sequencing adapter. Each 25 $\mu$L
 reaction contained 1x PCR Gold Buffer (Roche), 2.5 mM MgCl$_{2}$, 200 $\mu$M
 of each of the four dNTPs (Promega), 0.5 mg/mL BSA (New England Biolabs), 0.3
 $\mu$M of each primers, 1.25 U of Amplitaq Gold (Roche), and 8 $\mu$L of
 template. Template for each sample was added at normalized amounts in an
 attempt to prevent chimera formation, and each sample was amplified in
 triplicate. Thermal cycling occurred with an initial denaturation step of 5
 minutes at 95°C, followed by 40 cycles of amplification (20s at 95°C, 20s at
 53°C, 30s at 72°C), and a final extension step of 5 min at 72°C. Triplicate
 amplicons were pooled and purified using Agencourt AMPure PCR purification
 beads, following manufacturer’s protocol. Once cleaned, amplicons were
 quantified using PicoGreen nucleic acid quantification dyes (Molecular Probes)
 and pooled together in equimolar amounts. Samples were sent to the
 Environmental Genomics Core Facility at the University of South Carolina (now
 Selah Genomics) to be run on a Roche FLX 454 pyrosequencing machine. 

\subsection{Data analysis} 

\subsubsection{Sequence quality control} Sequences were initially screened by
maximum expected errors at a specific read length threshold \citep{23955772}
which has been shown to be as effective as denoising 454 reads with respect to
removing pyrosequencing errors. Specifically, reads were first truncated to 230
nucleotides (nt) (all reads shorter than 230 nt were discarded) and any read that
exceeded a maximum expected error threshold of 1.0 was removed. After
truncation and max
expected error trimming, 91\% of original reads remained. The first 30 nt
representing the forward primer and barcode on high quality, truncated reads
were trimmed. Remaining reads were taxonomically annotated using the "UClust"
taxonomic annotation framework in the QIIME software package \citep{20383131,
20709691} with cluster seeds from Silva SSU rRNA database \citep{17947321} 97\%
sequence identity OTUs as reference (release 111Ref). Reads annotated as
"Chloroplast", "Eukaryota", "Archaea", "Unassigned" or "mitochondria" were
culled from the dataset. Finally, reads were aligned to the Silva reference
alignment provided by the Mothur software package \citep{19801464} using the
Mothur NAST aligner \citep{16845035}. All reads that did not appear to align to
the expected amplicon region of the SSU rRNA gene were discarded. Quality
control parameters removed 34716 of 258763 raw reads.

\subsubsection{Sequence clustering}
Sequences were distributed into OTUs using the UParse methodology
\citep{23955772}. Specifically, cluster seeds were identified using USearch with
a collection of non-redundant reads sorted by count as input. The sequence
identity threshold for establishing a new OTU centroid was 97\%. After initial
cluster centroid selection, select 16S rRNA sequences trimmed to the same 16S
position as the other centroids from \citet{Yeager} were added to the centroid
collection. Specifically, \citet{Yeager} Colorado Plateau or Moab, Utah
sequences were added which included the 16S sequences for \textit{Calothrix
MCC-3A, Nostoc commune MCT-1, Nostoc commune MFG-1, Scytonema hyalinum DC-A,
Scytonema hyalinum FGP-7A, Spirirestis rafaelensis LQ-10}. Centroid sequences
that matched selected \citet{Yeager} sequences with greater than to 97\%
sequence identity were subsequently removed from the centroid collection. With
USearch/UParse, potential chimeras are identified during OTU centroid selection
and are not allowed to become cluster centroids effectively removing chimeras
from the read pool. All quality controlled reads were then mapped to cluster
centroids at an identity threshold of 97\% again using USearch. 95.6\% of
quality controlled reads could be mapped to centroids. Unmapped reads do not
count towards sample counts and are essentially removed from downstream
analyses. The USearch software version for cluster generation was 7.0.1090.

\begin{table}

\textbf{
\refstepcounter{table}
\label{table:yeager_2006} 
Table \arabic{table}. }{Chosen 16S sequences for strains in \citet{Yeager} included as OTU centroids}

{\begin{tabular}{ l l }
\toprule
\textbf{Accession of representative 16S rRNA seqeunce} & \textbf{Species Name} \\ \midrule 
DQ531701.1 & Scytonema hyalinum DC-A \\ \midrule 
DQ531697.1 & Scytonema hyalinum FGP-7A \\ \midrule
DQ531696.1 & Spirirestis rafaelensis LQ-10 \\ \midrule
DQ531703.1 & Nostoc commune MCT-1 \\ \midrule
DQ531699.1 & Nostoc commune MFG-1 \\ DQ531700.1 & Calothrix MCC-3A  \\ 
\bottomrule
\end{tabular}}{} 

\end{table}

%\begin{table}
%\begin{center}
%\begin{tabular}{ l l }
%\textbf{Accession of representative 16S rRNA seqeunce} &
%\textbf{Species Name} \\ 
%\hline
%DQ531701.1 & Scytonema hyalinum DC-A \\
%DQ531697.1 & Scytonema hyalinum FGP-7A \\
%DQ531696.1 & Spirirestis rafaelensis LQ-10 \\
%DQ531703.1 & Nostoc commune MCT-1 \\
%DQ531699.1 & Nostoc commune MFG-1 \\
%DQ531700.1 & Calothrix MCC-3A  \\
%\hline
%\end{tabular}
%\end{center}
%\caption{Chosen 16S sequences for strains in \citet{Yeager} included as OTU centroids}
%\label{table:yeager_2006}
%\end{table}

\subsubsection{Merging data from this study, \citet{Garcia_Pichel_2013}, and
\citet{Steven_2013}} As only sequences without corresponding quality scores
were publicly available from \citet{Garcia_Pichel_2013} and
\citet{Steven_2013}, these data sets were only quality screened by determining
if they covered the expected region of the 16S gene (described above). All data
(this study, \citet{Garcia_Pichel_2013} and \citet{Steven_2013}) were included
as input to USearch for OTU centroid selection and subsequent mapping to OTU
centroids. 

\subsubsection{Phylogenetic tree}
The alignment for the "\textit{Clostridiaceae}" phylogeny was created using
SSU-Align which is based on Infernal \citep{24008419, 19307242}. Columns in the
alignment that were not included in the SSU-Align covariance models or were
aligned with poor confidence (less than 95\% of characters in a position had
posterior probability alignment scores of at least 95\%) were masked for
phylogenetic reconstruction. Additionally, the alignment was trimmed to
coordinates such that all sequences in the alignment began and ended at the
same positions. The "\textit{Clostridiaceae}" tree included all top BLAST hits
(parameters below) for $^{15}$N \textit{Clostridiaceae} responders in the
Living Tree Project database \citep{Yarza_2008} in addition to BLAST hits
within a sequence identity
threshold of 97\% to $^{15}$N responders from the Silva SSURef\_NR SSU rRNA
database \citep{17947321}. Only one SSURef\_NR115 hit per study per OTU
("study" was determined by "title" field) was selected for the tree. FastTree
\citep{20224823} was used to build the tree and support values are
SH-like scores reported by FastTree.

\paragraph{Placement of short sequences into backbone phylogeny}
Short sequences were mapped to the reference backbone using pplacer
\citep{Matsen_2010} (default parameters). pplacer finds the edge placements
that maximize phylogenetic likelihood. Prior to being mapped to the reference
tree, short sequences were aligned to the reference alignment using Infernal
\citep{19307242} against the same SSU-Align covariance model used to align
reference sequences.

\subsubsection{BLAST searches}
BLAST searches were done with the "blastn" program from BLAST+ toolkit
\citep{20003500} version 2.2.29+. Default parameters were always employed and
the BioPython \citep{19304878} BLAST+ wrapper was used to invoke the blastn
program. Pandas \citep{citeulike:11241428} and dplyr \citep{dplyr} were used to
parse and munge BLAST output tables.  

\subsubsection{Identifying OTUs that incorporated $^{15}$N into their DNA}
SIP is a culture-independent approach towards defining identity-function
connections in microbial communities \citep{Buckley_2011, 17446886}. Microbes
incubated in the presence of $^{13}$C or $^{15}$N labeled substrates can
incorporate the stable heavy isotope into biomass if they participate in the
substrate's transformation.  Stable isotope labeled nucleic acids can then be
separated from unlabeled by buoyant density in a CsCl gradient. As the
buoyant density of a macromolecule is dependent on many factors in addition
to stable isotope incorporation (e.g.  GC-content in nucleic acids
\citep{25139123}), labeled nucleic acids from one microbial population may
have the same buoyant density of unlabeled nucleic acids from another (i.e.
each population's nucleic acids would be found at the same point along a
density gradient although only one population's nucleic acids are labeled).
Therefore it is imperative to compare density gradients with nucleic acids
from heavy stable isotope incubations to gradients from ``control''
incubations where everything mimics the experimental conditions except that
unlabeled substrates are used (and all DNA would be unlabeled).  By
contrasting "heavy" density gradient fractions in experimental density
gradients (hereafter referred to as "labeled" gradients) against heavy
fractions in control gradients, the identities of microbes with labeled
nucleic acids can be determined 

We used an RNA-Seq differential expression statistical framework
\citep{Love_2014} to find OTUs enriched in heavy fractions of labeled
gradients relative to corresponding density fractions in control gradients
(for review of RNA-Seq differential expression statistics applied to
microbiome OTU count data see \citet{24699258}). We use the term
“differential abundance” (coined by \citet{24699258}) to denote OTUs that
have different proportion means across sample classes (in this case the only
sample class is labeled/control).  CsCl gradient fractions were categorized
as "heavy" or "light". The heavy category denotes fractions with density
values above 1.725 g/mL. Since we are only interested in enriched OTUs
(labeled versus control), we used a one-sided z-test for differential
abundance (the null hypothesis is the labeled:control proportion mean ratio
for an OTU is less than a selected threshold). P-values were corrected with
the Benjamini and Hochberg method \citep{citeulike:1042553}. We selected a
log$_{2}$ fold change null threshold of 0.25 (or a labeled:control proportion
mean ratio of 1.19). DESeq2 was used to calculate the moderated log$_{2}$
fold change of labeled:control proportion mean ratios and corresponding
standard errors. Mean ratio moderation allows for reliable ratio ranking such
that high variance and likely statistically insignificant mean ratios are
appropriately shrunk and subsequently ranked lower than they would be as raw
ratios. To summarize, OTUs with high moderated labeled:control proportion
mean ratios have higher proportion means in heavy fractions of labeled
gradients relative to heavy fractions of control gradients, and therefore
have likely incorporated $^{15}$N into their DNA during the incubation.

\subsubsection{Ordination}
Principal coordinate ordinations depict the relationship between samples at
each time point (day 2 and 4). Bray-Curtis distances were used as the sample
distance metric for ordination. The Phyloseq \citep{24699258} wrapper for Vegan
\citep{vegan} (both R packages) was used to compute sample values along
principal coordinate axes. GGplot2 \citep{ggplot2} was used to display sample
points along the first and second principal axes.  

\subsubsection{Differential abundance in environmental samples}
Significance of OTU proportion mean differences with mean annual temperature
(for \citet{Garcia_Pichel_2013} data) and sample type (``BSC'' or ``below crust''
\citet{Steven_2013} data) was determined using the DESeq2 framework
\citep{24699258, Love_2014}. A sparsity threshold of 0.40 was set to screen out
sparse OTUs. No p-value correction was done for differential abundance in 
environmental samples as only six OTUs were considered for any test.

\subsection{Richness analyses} Rarefaction curves were created using
bioinformatics modules in the PyCogent Python package \citep{Knight_2007}.
Parametric richness estimates were made with CatchAll using only the best model
for total OTU estimates \citep{BUNGE_2010}.
